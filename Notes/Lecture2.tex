%%%%%%%%%%%%%%%%%%%%%%%%%%%%%%%%%%%%%%%%%%%%%%%%%%%%%%%%%%%%%%%%%%%%%%%%%%%%%%%%%%%%%%%%%%%%%%%%%%%%%%%%%%%%%%%%%%%%%%%%%%%%%%%%%%%%%%%%%%%%%%%%%%%%%%%%%%%%%%%%%%%
% Written By Michael Brodskiy
% Class: Analysis of Random Phenomena
% Professor: I. Salama
%%%%%%%%%%%%%%%%%%%%%%%%%%%%%%%%%%%%%%%%%%%%%%%%%%%%%%%%%%%%%%%%%%%%%%%%%%%%%%%%%%%%%%%%%%%%%%%%%%%%%%%%%%%%%%%%%%%%%%%%%%%%%%%%%%%%%%%%%%%%%%%%%%%%%%%%%%%%%%%%%%%

\documentclass[12pt]{article} 
\usepackage{alphalph}
\usepackage[utf8]{inputenc}
\usepackage[russian,english]{babel}
\usepackage{titling}
\usepackage{float}
\usepackage{amsmath}
\usepackage{graphicx}
\usepackage{enumitem}
\usepackage{amssymb}
\usepackage[super]{nth}
\usepackage{everysel}
\usepackage{ragged2e}
\usepackage{geometry}
\usepackage{multicol}
\usepackage{fancyhdr}
\usepackage{cancel}
\usepackage{siunitx}
\usepackage{physics}
\usepackage{tikz}
\usepackage{mathdots}
\usepackage{yhmath}
\usepackage{cancel}
\usepackage{color}
\usepackage{xcolor}
\usepackage{colortbl}
\usepackage{array}
\usepackage{multirow}
\usepackage{gensymb}
\usepackage{tabularx}
\usepackage{extarrows}
\usepackage{booktabs}
\usepackage{lastpage}
\usetikzlibrary{fadings}
\usetikzlibrary{patterns}
\usetikzlibrary{shadows.blur}
\usetikzlibrary{shapes}

\geometry{top=1.0in,bottom=1.0in,left=1.0in,right=1.0in}
\newcommand{\subtitle}[1]{%
  \posttitle{%
    \par\end{center}
    \begin{center}\large#1\end{center}
    \vskip0.5em}%

}
\usepackage{hyperref}
\hypersetup{
colorlinks=true,
linkcolor=blue,
filecolor=magenta,      
urlcolor=blue,
citecolor=blue,
}


\title{Lecture 2 — Random Variables}
\date{\today}
\author{Michael Brodskiy\\ \small Professor: I. Salama}

\begin{document}

\maketitle

\begin{itemize}

  \item A random variable is a function that maps the outcomes of a random experiment into a set of real numbers

  \item The Probability Mass Function (PMF)

    \begin{itemize}

      \item The PMF is a probability measure that gives us probabilities of the possible values for a random variable

      \item The PMF may be defined as:

        $$P_x(x)=\left\{ \begin{array}{ll} P(X=x), & \text{if }x\in S_x\\ 0, & \text{ Otherwise}\end{array}$$

      \item The probability mass function can be obtained using the probabilities of the corresponding sample space outcomes

    \end{itemize}

  \item The Bernoulli Random Variable

    \begin{itemize}

      \item We may write this as $X=\text{Bernoulli}(p)$

      \item This can be expressed as:

        $$P_x(x)=\left\{ \begin{array}{ll} 1-p,&x=0\\p,&x=1\\0,&\text{otherwise}\end{array}$$

    \end{itemize}

  \item The Binomial Random Variable

    \begin{itemize}

      \item We may write this as $X=\text{Binomial}(n,p)$

      \item This can be expressed as:

        $$P_x(k)=\left\{ \begin{array}{ll} \left( \begin{matrix} n\\k\end{matrix} \right)p^k(1-p)^{n-k},&0\leq k\leq n\\0,&\text{otherwise}\end{array}$$

    \end{itemize}

  \item The Geometric Random Variable

    \begin{itemize}

      \item We may write this as: $X=\text{Geometric}(p)$

      \item This can be expressed as:

        $$P_x(x)=\left\{ \begin{array}{ll} p(1-p)^{x-1},&x=1,2,3\cdots\\0,&\text{otherwise}\end{array}$$

    \end{itemize}

  \item The Poisson Random Variable

    \begin{itemize}

      \item We may write this as: $X=\text{Poisson}(a)$

      \item This can be expressed as:

        $$P_K(k)=\left\{ \begin{array}{ll} \frac{a^k}{k!}e^{-a},&k=0,1,2,3\cdots\\0,&\text{otherwise}\end{array}$$

        \item The Poisson random variable can be described using the average arrival rate $a=\lambda T$

    \end{itemize}

  \item The Pascal or Negative Binomial Random Variable

    \begin{itemize}

      \item We may write this as: $X=\text{Pascal}(k,p)$

      \item This can be expressed as:

        $$P_X(x)=\left\{ \begin{array}{ll} \left( \begin{matrix} x-1\\k-1\end{matrix} \right)p^k(1-p)^{x-k},&x=k,k+1,\cdots\\0,&\text{otherwise}\end{array}$$

    \end{itemize}

  \item The Discrete Uniform Random Variable

    \begin{itemize}

      \item We may write this as: $N=\text{Uniform}(k,l)$

      \item This can be expressed as:

        $$P_N(n)=\left\{ \begin{array}{ll} \frac{1}{l-k-1},&l-k=2,3,4,\cdots\\0,&\text{otherwise}\end{array}$$

    \end{itemize}

\end{itemize}

\end{document}

