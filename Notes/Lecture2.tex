%%%%%%%%%%%%%%%%%%%%%%%%%%%%%%%%%%%%%%%%%%%%%%%%%%%%%%%%%%%%%%%%%%%%%%%%%%%%%%%%%%%%%%%%%%%%%%%%%%%%%%%%%%%%%%%%%%%%%%%%%%%%%%%%%%%%%%%%%%%%%%%%%%%%%%%%%%%%%%%%%%%
% Written By Michael Brodskiy
% Class: Analysis of Random Phenomena
% Professor: I. Salama
%%%%%%%%%%%%%%%%%%%%%%%%%%%%%%%%%%%%%%%%%%%%%%%%%%%%%%%%%%%%%%%%%%%%%%%%%%%%%%%%%%%%%%%%%%%%%%%%%%%%%%%%%%%%%%%%%%%%%%%%%%%%%%%%%%%%%%%%%%%%%%%%%%%%%%%%%%%%%%%%%%%

\include{Includes.tex}

\title{Lecture 2 — Random Variables}
\date{\today}
\author{Michael Brodskiy\\ \small Professor: I. Salama}

\begin{document}

\maketitle

\begin{itemize}

  \item A random variable is a function that maps the outcomes of a random experiment into a set of real numbers

  \item The Probability Mass Function (PMF)

    \begin{itemize}

      \item The PMF is a probability measure that gives us probabilities of the possible values for a random variable

      \item The PMF may be defined as:

        $$P_x(x)=\left\{ \begin{array}{ll} P(X=x), & \text{if }x\in S_x\\ 0, & \text{ Otherwise}\end{array}$$

      \item The probability mass function can be obtained using the probabilities of the corresponding sample space outcomes

    \end{itemize}

\end{itemize}

\end{document}

