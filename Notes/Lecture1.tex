%%%%%%%%%%%%%%%%%%%%%%%%%%%%%%%%%%%%%%%%%%%%%%%%%%%%%%%%%%%%%%%%%%%%%%%%%%%%%%%%%%%%%%%%%%%%%%%%%%%%%%%%%%%%%%%%%%%%%%%%%%%%%%%%%%%%%%%%%%%%%%%%%%%%%%%%%%%%%%%%%%%
% Written By Michael Brodskiy
% Class: Analysis of Random Phenomena
% Professor: I. Salama
%%%%%%%%%%%%%%%%%%%%%%%%%%%%%%%%%%%%%%%%%%%%%%%%%%%%%%%%%%%%%%%%%%%%%%%%%%%%%%%%%%%%%%%%%%%%%%%%%%%%%%%%%%%%%%%%%%%%%%%%%%%%%%%%%%%%%%%%%%%%%%%%%%%%%%%%%%%%%%%%%%%

\documentclass[12pt]{article} 
\usepackage{alphalph}
\usepackage[utf8]{inputenc}
\usepackage[russian,english]{babel}
\usepackage{titling}
\usepackage{float}
\usepackage{amsmath}
\usepackage{graphicx}
\usepackage{enumitem}
\usepackage{amssymb}
\usepackage[super]{nth}
\usepackage{everysel}
\usepackage{ragged2e}
\usepackage{geometry}
\usepackage{multicol}
\usepackage{fancyhdr}
\usepackage{cancel}
\usepackage{siunitx}
\usepackage{physics}
\usepackage{tikz}
\usepackage{mathdots}
\usepackage{yhmath}
\usepackage{cancel}
\usepackage{color}
\usepackage{xcolor}
\usepackage{colortbl}
\usepackage{array}
\usepackage{multirow}
\usepackage{gensymb}
\usepackage{tabularx}
\usepackage{extarrows}
\usepackage{booktabs}
\usepackage{lastpage}
\usetikzlibrary{fadings}
\usetikzlibrary{patterns}
\usetikzlibrary{shadows.blur}
\usetikzlibrary{shapes}

\geometry{top=1.0in,bottom=1.0in,left=1.0in,right=1.0in}
\newcommand{\subtitle}[1]{%
  \posttitle{%
    \par\end{center}
    \begin{center}\large#1\end{center}
    \vskip0.5em}%

}
\usepackage{hyperref}
\hypersetup{
colorlinks=true,
linkcolor=blue,
filecolor=magenta,      
urlcolor=blue,
citecolor=blue,
}


\title{Lecture 1 — Basics of Probability Theory}
\date{\today}
\author{Michael Brodskiy\\ \small Professor: I. Salama}

\begin{document}

\maketitle

\begin{itemize}

  \item What is a Set?

    \begin{itemize}

      \item A set is a collection of objects (elements) that make up the set

      \item We usually use upper case letters to describe a set and lower-case letters to refer to the elements

      \item A set can be defined using enumerations:

        $$A=\left\{ \text{Jane}, \text{Bill},\cdots \right\}$$
        $$B=\left\{ 1,2,3,\cdots \right\}$$

      \item A set can also be defined using a description method

      \item $A=\left\{ x\,|\,x\text{ satisfies some property} \right\}$

      \item For example:

        $$A=\left\{ \text{Students } | \text{ Students who earned an 'A'} \right\}$$

      \item A set can have a finite or infinite number of elements

      \item Useful notations:

        \begin{itemize}

          \item $x\in A \equiv$ element $x$ is contained in $A$

          \item $x\notin A \equiv$ element $x$ is \underline{not} contained in $A$

          \item $C=\left\{  \right\}=\emptyset$ — $C$ is an empty or null set

          \item $D=S$ — Universal set including all elements in a given category

          \item $A\subset B$ — $A$ is a subset of set $B$

          \item Simple Set: A set with a single element
            
          \item Set equality: $A=B$ only if $A\subset B$ and $B\subset A$

          \item $A^C\equiv$ complement of set $A$, which includes all elements in a given category that are not in set $A$

        \end{itemize}

    \end{itemize}

  \item The Inclusion-Exclusion Rule

    \begin{itemize}

      \item For two finite sets $A$ and $B$, we have:

        $$|A\cup B|=|A|+|B|-|A\cap B|$$

      \item This can be expanded to three sets (with a new set $C$):

        $$|A\cup B\cup C|=|A|+|B|+|C|-|A\cap B| - |A\cap C| - |B\cap C| + |A\cap B\cap C|$$

    \end{itemize}

  \item Types of Sets

    \begin{itemize}

      \item Collectively Exhaustive Sets — A collection of sets $A_1,A_2,\cdots,A_n$ are collectively exhaustive if (at least one of the events must occur):

        $$A_1\cup A_2 \cup \cdots \cup A_n=S$$

      \item Mutually Exclusive Sets — Two sets are mutually exclusive if they have no elements in common:

        $$A_i\cap A_f = \emptyset\quad i\neq j$$

      \item Partitions — A collection of sets $A_1,A_2,\cdots,A_n$ is a partition if they are both mutually exclusive and collectively exhaustive:

        $$A_1\cup A_2\cup \cdots \cup A_n=S\quad\text{ and }\quad A_i\cap A_j=\emptyset\quad i\neq j$$

    \end{itemize}

  \item Algebraic Rules of Manipulating Sets

    \begin{itemize}

      \item Commutative: $A\cap B= B\cap A$ and $A\cup B = B\cup A$

      \item Associative: $A\cap (B\cap C)= (A\cap B)\cap C$

      \item Distributive: $A\cup (B\cap C)= (A\cup B)\cap (A\cup C)$

      \item De Morgan's Law: $(A\cup B)^c = A^c\cap B^c$ and $(A\cap B)^c=A^c\cup B^c$

    \end{itemize}

  \item Applying Set Theory to Probability Theory

    \begin{itemize}

      \item The Probability Function

        \begin{itemize}

          \item A probability measure $P[x]$ is a function that maps events in the sample space to real numbers

          \item The probability function assigns a value $0$ to $1$ to a certain event; for example, the probability function of rolling a single, standard die would be $P[1]=P[2]=\cdots=P[6]=1/6$

        \end{itemize}

      \item Axioms of Probability

        \begin{itemize}

          \item For any event $A$, $P[A]\geq 0$

          \item $P[S]=1$

          \item For any countable collection of mutually exclusive events:

            $$P[A_1\cup A_2\cup\cdots\cup A_n]=P[A_1]+P[A_2]+\cdots+P[A_n]$$

        \end{itemize}

      \item Consequences of the Axioms

        \begin{itemize}

          \item For mutually exclusive events, $P[A\cup B]=P[A]+P[B]$

          \item For mutually exclusive events, $A_1, A_2, \cdots, A_N$:

            $$P[A_1\cup A_2\cup\cdots\cup A_N]=P[A_1]+P[A_2]+\cdots+P[A_N]$$

        \end{itemize}

    \end{itemize}

  \item Prior, Posterior, and Conditional Probabilities

    \begin{itemize}

      \item Prior Probability $P[A]$: The probability of an event $A$ before any other information or evidence is considered

      \item Conditional Probability $P[A|B]$: The probability of $A$ given that $B$ has occurred, showing how $B$ influences the likelihood of $A$

        \begin{itemize}

          \item $P[A]=$ Prior Probability of Event $A\to P[A]=\dfrac{|A|}{|S|}$

          \item $P[A|B]=$ Conditional Probability of $A$ Given Event $B$ Occurred

          \item When event $B$ occurs, the sample space is reduced to $B$

            $$P[A|B]=\frac{|A\cap B|}{|B|}=\frac{P[A\cap B]}{P[B]}$$

        \end{itemize}

      \item Posterior Probability $P[B|A]$: The probability of event $B$ after observing event $A$. It represents a ``revised belief'' about $B$, incorporating the new information provided by $A$. Posterior probability adjusts the prior probability of $B$ based on the evidence from $A$

    \end{itemize}

  \item Bayes Rule and Statistical Independence

    $$P[A\cap B]=P[AB]=P[B]P[A|B]=P[A]P[B|A]$$

    \begin{itemize}

      \item Bayes' Rule states:

        $$P[A|B]=\frac{P[AB]}{P[B]}\quad\text{ and }\quad P[B|A]=\frac{P[AB]}{P[A]}$$

      \item $A$ and $B$ are statistically independent if:

        $$P[AB]=P[A]P[B]\quad\text{ or }\quad P[A|B]=P[A]\text{ and }P[B|A]=P[B]$$

    \end{itemize}

  \item Probability Chain Rule of Three Events

    $$P[ABC]=P[C]P[AB|C]=P[C]P[B|C]P[A|BC]$$

    \begin{itemize}

      \item $A,B,$ and $C$ are statistically independent if $P[ABC]=P[A]P[B]P[C]$ and each pair satisfies the condition of independence:

        \begin{itemize}

          \item $A$ and $B$ are statistically independent

          \item $A$ and $C$ are statistically independent

          \item $B$ and $C$ are statistically independent

        \end{itemize}

    \end{itemize}

  \item Permutations

    \begin{itemize}
        
      \item The number of ways to arrange $k$ distinguishable objects if a set of $n$ distinguishable objects with repetition is:

        $$\#=n^k$$

    \end{itemize}

  \item Combinations

    \begin{itemize}

      \item Given a set of $n$ distinct objects, any unordered subset $k$ of the objects is called a combination of size $k$. The number of combinations of size $k$ that can be formed from $n$ distinct objects will be denoted by $_nC_k$ or $\left( \begin{matrix} n\\k\end{matrix} \right)$ and is given by:

        $$\frac{n!}{k!(n-k)!}$$

    \end{itemize}

  \item Binomial Probability

    \begin{itemize}

      \item The probability of having exactly $n_1$ successes and $n_o$ failures in $n$ independent trials is given by:

        $$p_b=\left( \begin{matrix} n\\n_1\end{matrix} \right)p^{n_1}(1-p)^{n-n_1}$$

      \item $n_o+n_1=n$

      \item $p$ is the probability of success in any trial independent of any other trial

    \end{itemize}

  \item Multinomial Probability

    \begin{itemize}
        
      \item In $n$ independent repetitions of an experiment with sample space $\left[ s_1,s_2,\cdots,s_n \right]$ and the corresponding probabilities $\left[ p_1,p_2,\cdots,p_n \right]$, the probability of having exactly $n_i$ outcomes of type $s_i$ is given by:

        $$p=\frac{n!}{n_1!n_2!\cdots!n_m!}p_1^{n_1}p_2^{n_2}\cdots p_m^{n_m}\quad\text{ where }n=n_1+n_2+\cdots+n_m$$

    \end{itemize}

\end{itemize}

\end{document}

