%%%%%%%%%%%%%%%%%%%%%%%%%%%%%%%%%%%%%%%%%%%%%%%%%%%%%%%%%%%%%%%%%%%%%%%%%%%%%%%%%%%%%%%%%%%%%%%%%%%%%%%%%%%%%%%%%%%%%%%%%%%%%%%%%%%%%%%%%%%%%%%%%%%%%%%%%%%%%%%%%%%
% Written By Michael Brodskiy
% Class: Analysis of Random Phenomena
% Professor: I. Salama
%%%%%%%%%%%%%%%%%%%%%%%%%%%%%%%%%%%%%%%%%%%%%%%%%%%%%%%%%%%%%%%%%%%%%%%%%%%%%%%%%%%%%%%%%%%%%%%%%%%%%%%%%%%%%%%%%%%%%%%%%%%%%%%%%%%%%%%%%%%%%%%%%%%%%%%%%%%%%%%%%%%

\include{Includes.tex}

\title{Homework 3}
\date{\today}
\author{Michael Brodskiy\\ \small Professor: I. Salama}

\begin{document}

\maketitle

\begin{enumerate}

  \item

    \begin{enumerate}

      \item We want to find the probability that both units failed. This can be expressed as event $A$ AND event $B$, indicating failure of the processors. Per our formulas, we know:

        $$P[A|B]=\frac{P[A\cap B]}{P[B]}$$

        Per the given information, we know that $P[A|B]=.3$ and $P[B]=.02$. Thus, we may calculate to get:

        $$\boxed{P[A\cap B]=.006}$$

      \item We may calculate the probability of no good units in five systems by simply raising part (a) to the power of 5 to get:

        $$\boxed{P_5[A\cap B]=7.776\cdot10^{-12}}$$

      \item The probability that there is a working unit in a single computer can be given by the complement of no working units. Thus, we see:

        $$P[(A\cap B)^c]=1-.006=.994$$

        We can then sum the possible combinations for the other two computers (at least one working or none working) to get:

        $$P_3[(A\cap B)^c]=.994(.994^2+(.994)(.006)+.006^2)$$
        $$\boxed{P_3[(A\cap B)^c]=.9881}$$

    \end{enumerate}

  \item

    \begin{enumerate}

      \item The probability that the system does not fail may be expressed as:

        $$P[F^c]=(1-p)^{20}$$

        We can thus take the complement and say that failure will occur for:

        $$P[F]=1-(1-p)^{20}$$

        From the problem, the system failure rate is 20\%, which gives us:

        $$1-(1-p)^{20}=.2$$

        From here, we solve:

        $$(1-p)^{20}=.8$$
        $$1-p=\sqrt[20]{.8}$$
        $$p=1-\sqrt[20]{.8}$$
        $$\boxed{p=.011095}$$

      \item Similar to part (a), we now solve for a failure rate of $.1$:

        $$1-(1-p)^{20}=.1$$
        $$(1-p)^{20}=.9$$
        $$p=1-\sqrt[20]{.9}$$
        $$\boxed{p=.0052542}$$

    \end{enumerate}

    \setcounter{enumi}{3}

  \item

    \begin{enumerate}

      \item We can calculate this by multiplying the success rates together:

        $$P_3=(.9)^2(.8)$$
        $$\boxed{P_3=.648}$$

      \item We want to calculate the probability of at least one of the components succeeding. This may be expressed as:

        $$P_{1+}=\underbrace{2[(.9)(.1)(.2)]}_{\text{One router}}+\overbrace{[(.1)^2(.8)]}^{\text{Only switch}}+\underbrace{[(.9)^2(.2)]}_{\text{Both routers}}+\overbrace{2[(.9)(.8)(.1)]}^{\text{One router/one switch}}+.648$$
        $$\boxed{P_{1+}=.998}$$

      \item To find the odds of exactly one of the components passing, we simply subtract cases from above in which multiple devices pass. This gives us:

        $$P_{1}=.998-2[(.9)(.8)(.1)]-[(.9)^2(.2)]-.648$$
        $$\boxed{P_{1}=.044}$$

      \item Since we know the probability that no components fail, we simply take the complement to find the probability that a component does fail:

        $$P_3^c=1-.648$$
        $$\boxed{P_3^c=.352}$$

      \item Per our formulas, we can find $P[\text{all}|\text{one}]$ as:

        $$P[\text{all}|\text{one}]=\frac{.648}{.998}$$
        $$\boxed{P[\text{all}|\text{one}]=.6493}$$

    \end{enumerate}

  \item

    \begin{enumerate}

      \item 

      \item 

      \item 

      \item 

    \end{enumerate}

  \item

    \begin{enumerate}
        
      \item 

      \item 

    \end{enumerate}

  \item

  \item

  \item

    \begin{enumerate}

      \item 

      \item 

      \item 

      \item 

    \end{enumerate}

  \item

    \begin{enumerate}

      \item 

      \item 

      \item 

    \end{enumerate}

\end{enumerate}

\end{document}

