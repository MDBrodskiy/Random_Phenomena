%%%%%%%%%%%%%%%%%%%%%%%%%%%%%%%%%%%%%%%%%%%%%%%%%%%%%%%%%%%%%%%%%%%%%%%%%%%%%%%%%%%%%%%%%%%%%%%%%%%%%%%%%%%%%%%%%%%%%%%%%%%%%%%%%%%%%%%%%%%%%%%%%%%%%%%%%%%%%%%%%%%
% Written By Michael Brodskiy
% Class: Analysis of Random Phenomena
% Professor: I. Salama
%%%%%%%%%%%%%%%%%%%%%%%%%%%%%%%%%%%%%%%%%%%%%%%%%%%%%%%%%%%%%%%%%%%%%%%%%%%%%%%%%%%%%%%%%%%%%%%%%%%%%%%%%%%%%%%%%%%%%%%%%%%%%%%%%%%%%%%%%%%%%%%%%%%%%%%%%%%%%%%%%%%

\include{Includes.tex}

\title{Homework 3}
\date{\today}
\author{Michael Brodskiy\\ \small Professor: I. Salama}

\begin{document}

\maketitle

\begin{enumerate}

  \item

    \begin{enumerate}

      \item We want to find the probability that both units failed. This can be expressed as event $A$ AND event $B$, indicating failure of the processors. Per our formulas, we know:

        $$P[A|B]=\frac{P[A\cap B]}{P[B]}$$

        Per the given information, we know that $P[A|B]=.3$ and $P[B]=.02$. Thus, we may calculate to get:

        $$\boxed{P[A\cap B]=.006}$$

      \item We may calculate the probability of no good units in five systems by simply raising part (a) to the power of 5 to get:

        $$\boxed{P_5[A\cap B]=7.776\cdot10^{-12}}$$

      \item The probability that there is a working unit in a single computer can be given by the complement of no working units. Thus, we see:

        $$P[(A\cap B)^c]=1-.006=.994$$

        We can then sum the possible combinations for the other two computers (at least one working or none working) to get:

        $$P_3[(A\cap B)^c]=.994(.994^2+(.994)(.006)+.006^2)$$
        $$\boxed{P_3[(A\cap B)^c]=.9881}$$

    \end{enumerate}

  \item

    \begin{enumerate}

      \item The probability that the system does not fail may be expressed as:

        $$P[F^c]=(1-p)^{20}$$

        We can thus take the complement and say that failure will occur for:

        $$P[F]=1-(1-p)^{20}$$

        From the problem, the system failure rate is 20\%, which gives us:

        $$1-(1-p)^{20}=.2$$

        From here, we solve:

        $$(1-p)^{20}=.8$$
        $$1-p=\sqrt[20]{.8}$$
        $$p=1-\sqrt[20]{.8}$$
        $$\boxed{p=.011095}$$

      \item Similar to part (a), we now solve for a failure rate of $.1$:

        $$1-(1-p)^{20}=.1$$
        $$(1-p)^{20}=.9$$
        $$p=1-\sqrt[20]{.9}$$
        $$\boxed{p=.0052542}$$

    \end{enumerate}

    \setcounter{enumi}{3}

  \item

    \begin{enumerate}

      \item We can calculate this by multiplying the success rates together:

        $$P_3=(.9)^2(.8)$$
        $$\boxed{P_3=.648}$$

      \item We want to calculate the probability of at least one of the components succeeding. This may be expressed as:

        $$P_{1+}=\underbrace{2[(.9)(.1)(.2)]}_{\text{One router}}+\overbrace{[(.1)^2(.8)]}^{\text{Only switch}}+\underbrace{[(.9)^2(.2)]}_{\text{Both routers}}+\overbrace{2[(.9)(.8)(.1)]}^{\text{One router/one switch}}+.648$$
        $$\boxed{P_{1+}=.998}$$

      \item To find the odds of exactly one of the components passing, we simply subtract cases from above in which multiple devices pass. This gives us:

        $$P_{1}=.998-2[(.9)(.8)(.1)]-[(.9)^2(.2)]-.648$$
        $$\boxed{P_{1}=.044}$$

      \item Since we know the probability that no components fail, we simply take the complement to find the probability that a component does fail:

        $$P_3^c=1-.648$$
        $$\boxed{P_3^c=.352}$$

      \item Per our formulas, we can find $P[\text{all}|\text{one}]$ as:

        $$P[\text{all}|\text{one}]=\frac{.648}{.998}$$
        $$\boxed{P[\text{all}|\text{one}]=.6493}$$

    \end{enumerate}

  \item

    \begin{enumerate}

      \item Given that $n$ can be the range of $[2,5]$, we know that the PMF must contain:

        $$\left\{ \frac{c}{4},\frac{c}{8},\frac{c}{16},\frac{c}{32} \right\}$$

        We know that the probabilities within a PMF must sum to 1, so we get:

        $$\frac{c}{4}+\frac{c}{8}+\frac{c}{16}+\frac{c}{32}=1$$

        We simplify to find:

        $$8c+4c+2c+c=32$$
        $$\boxed{c=\frac{32}{17}}$$

      \item We can take the complement of the probability that two packets are transmitted to find:

        $$P[3+]=1-\frac{1}{4}\left( \frac{32}{17} \right)$$
        $$\boxed{P[3+]=\frac{7}{17}=.4118}$$

      \item We can find the probability of an odd packet to be:

        $$P[3+\cap Odd]=\left( \frac{1}{8}+\frac{1}{32} \right)\frac{32}{17}$$
        $$P[3+\cap Odd]=\frac{5}{17}$$

        We then divide by the probability of three or more to find:

        $$\boxed{P[Odd|3+]=\frac{5}{7}=.7143}$$

      \item We can express the probability for a given $n$ as:

        $$P[C]=\sum_{n=2}^5 P_N(n)(.95)^n$$

        This gives us the following values:

        \begin{center}
          \begin{tabular}[H]{|c|c|}
           \hline
           $n$ & $P_n[C]$\\
           \hline
           2 & .4247\\
           \hline
           3 & .2017\\
           \hline
           4 & .095824\\
           \hline
           5 & .045517\\
           \hline
           Total & .7677\\
           \hline
          \end{tabular}
        \end{center}

    \end{enumerate}

  \item

    \begin{enumerate}
        
      \item Based on the given information, we may express the probability of 2 or 3 servers as $c$. This gives us:

        $$\left\{ \frac{c}{4},\frac{c}{2},c,c,\frac{c}{4} \right\}$$

        We know the values need to sum to 1, which gives us:

        $$\frac{2c}{4}+\frac{c}{2}+2c=1$$
        $$\frac{c}{2}+\frac{c}{2}+2c=1$$
        $$3c=1$$
        $$\boxed{c=\frac{1}{3}}$$

        Thus, we get:

        $$\left\{ \frac{1}{12},\frac{1}{6},\frac{1}{3},\frac{1}{3},\frac{1}{12} \right\}$$

      \item Based on the PMF above, we sum to get:

        $$P[2+]=1-\frac{1}{12}-\frac{1}{6}$$
        $$\boxed{P[2+]=.75}$$

    \end{enumerate}

  \item We may express the PMF of A as a binomial distribution, since the pulled resistor is either $A$ or not. This gives us $n=10$ and a probability of $p=1/3$, since each resistor is equally likely. This gives us the terms of the PMF as:

    $$\left( \begin{matrix} 10\\n\end{matrix} \right)(1/3)^{p}(2/3)^{p-n}$$

    We iterate to get:

    $$P_n(x)=\left\{.017342, .086708, .19509, .26012, .22761, .13656, .056902,$$$$.016258, .0030483, .00033870, .000016935 \}$$

    From here, we find the probability of 2 or more $A$ resistors as:

    $$P[2+]=1-.017342-.086708$$
    $$\boxed{P[2+]=.896}$$

  \item We begin by finding the probability that a generation-ending pair is combined. We may iterate through possible values of $S_M$ and $S_L$ and find that the pairs with absolute differences greater than three, in form $(l,m)$ are:

    $$(1,5),(1,8),(2,8),\text{ and }(3,8)$$

    We know that the total quantity of possible pairs can be found by multiplying together the quantity of possible values of $l$ and $m$, which gives:

    $$N=(4)(3)=12$$

    Thus, we find the probability that a generation-ending pair is found is:

    $$p=\frac{4}{12}=.33\bar{3}$$

    We can write the PMF as:

    $$\boxed{P_N(x)=\left( \frac{2}{3} \right)^{1-N}\left( \frac{1}{3} \right)}$$

    We can find the probability that $N=5$ as:

    $$P_5(x)=\left( \frac{2}{3} \right)^4\left( \frac{1}{3} \right)$$
    $$\boxed{P_5(x)=.065844}$$

    We may define the probability that $N>5$ as:

    $$P_{>5}(x)=1-\sum_{n=1}^{5} P_N(x)$$
    $$\boxed{P_{>5}(x)=.1317}$$

  \item

    \begin{enumerate}

      \item We can define the PMF as:

        $$\boxed{P_N(x)=\left( \begin{matrix} 20\\ N\end{matrix} \right)p^N(1-p)^{20-N}}$$

      \item Using the above, we take that $N\leq 15$ and $p=.8$ to calculate the probability of rejection as:

        $$P(\text{rejection}|p=.8)=\sum_{N=0}^{15}\left( \begin{matrix} 20\\ N\end{matrix} \right)(.8)^N(.2)^{20-N}$$

        We iterate to find:
        
        $$\boxed{P(\text{rejection}|p=.8)=.3704}$$

      \item Similar to part (b), we can find the probability that the claim is \underline{not} rejected given $p=.7$ by writing:

        $$P(\text{not rejected}|p=.7)=1-\sum_{N=0}^{15}\left( \begin{matrix} 20\\ N\end{matrix} \right)(.7)^N(.3)^{20-N}$$

        This gives us:

        $$\boxed{P(\text{not rejected}|p=.7)=.2375}$$

      \item We can see that (b) would decrease, since one less value (when $N=15$) is added to the probability, while the probability for (c) would increase, since this value is no longer subtracted. Thus, we find:

        $$P_{b,\text{new}}=.3704 - \left( \begin{matrix}20\\15\end{matrix} \right)(.8)^{15}(.2)^{5}$$
        $$\boxed{P_{b,\text{new}}=.1958}$$

        $$P_{c,\text{new}}=.2375 + \left( \begin{matrix}20\\15\end{matrix} \right)(.7)^{15}(.3)^{5}$$
        $$\boxed{P_{c,\text{new}}=.4164}$$

    \end{enumerate}

  \item

    \begin{enumerate}

      \item We may observe that the PMF is simply a geometric distribution and can be written as ($n\geq 1$):

        $$\boxed{P_N(x)=p^n(1-p)^{n-1}}$$

      \item We can calculate the probability as:

        $$P_3(x)=\sum_{n=1}^3 p^n(1-p)^{n-1}$$

        We can expand this to get:

        $$p+p(1-p)+p(1-p)^2\geq.95$$
        $$-.95+p+p(1-p)+p(1-p)^2\geq0$$

        Solving, we find that the minimal value of $p$ is:

        $$\boxed{p\geq .6316}$$

      \item Although the formula for the PMF remains the same, it is different in that $n$ is now capped, such that $1\leq n\leq 5$. With $p=.7$ we may write this as:

        $$\boxed{P_N(x)=(.7)^n(.3)^{n-1},\quad 1\leq n\leq 5}$$

        We can calculate the probability that delivery fails as:

        $$P_{>5}(x)=1-\sum_{n=1}^5 (.7)^n(.3)^{n-1}$$

        This gives us:

        $$\boxed{P_{>5}(x)=.1143}$$

    \end{enumerate}

\end{enumerate}

\end{document}

