%%%%%%%%%%%%%%%%%%%%%%%%%%%%%%%%%%%%%%%%%%%%%%%%%%%%%%%%%%%%%%%%%%%%%%%%%%%%%%%%%%%%%%%%%%%%%%%%%%%%%%%%%%%%%%%%%%%%%%%%%%%%%%%%%%%%%%%%%%%%%%%%%%%%%%%%%%%%%%%%%%%
% Written By Michael Brodskiy
% Class: Analysis of Random Phenomena
% Professor: I. Salama
%%%%%%%%%%%%%%%%%%%%%%%%%%%%%%%%%%%%%%%%%%%%%%%%%%%%%%%%%%%%%%%%%%%%%%%%%%%%%%%%%%%%%%%%%%%%%%%%%%%%%%%%%%%%%%%%%%%%%%%%%%%%%%%%%%%%%%%%%%%%%%%%%%%%%%%%%%%%%%%%%%%

\documentclass[12pt]{article} 
\usepackage{alphalph}
\usepackage[utf8]{inputenc}
\usepackage[russian,english]{babel}
\usepackage{titling}
\usepackage{float}
\usepackage{amsmath}
\usepackage{graphicx}
\usepackage{enumitem}
\usepackage{amssymb}
\usepackage[super]{nth}
\usepackage{everysel}
\usepackage{ragged2e}
\usepackage{geometry}
\usepackage{multicol}
\usepackage{fancyhdr}
\usepackage{cancel}
\usepackage{siunitx}
\usepackage{physics}
\usepackage{tikz}
\usepackage{mathdots}
\usepackage{yhmath}
\usepackage{cancel}
\usepackage{color}
\usepackage{xcolor}
\usepackage{colortbl}
\usepackage{array}
\usepackage{multirow}
\usepackage{gensymb}
\usepackage{tabularx}
\usepackage{extarrows}
\usepackage{booktabs}
\usepackage{lastpage}
\usetikzlibrary{fadings}
\usetikzlibrary{patterns}
\usetikzlibrary{shadows.blur}
\usetikzlibrary{shapes}

\geometry{top=1.0in,bottom=1.0in,left=1.0in,right=1.0in}
\newcommand{\subtitle}[1]{%
  \posttitle{%
    \par\end{center}
    \begin{center}\large#1\end{center}
    \vskip0.5em}%

}
\usepackage{hyperref}
\hypersetup{
colorlinks=true,
linkcolor=blue,
filecolor=magenta,      
urlcolor=blue,
citecolor=blue,
}


\title{Homework 5}
\date{\today}
\author{Michael Brodskiy\\ \small Professor: I. Salama}

\begin{document}

\maketitle

\begin{enumerate}

  \item

    \begin{enumerate}

      \item To be a valid CDF, we know that the terms continuously build until they sum to 1. In this case, all of the terms become 1 at $v=10$. Thus, we can differentiate to find the PDF:

        $$f_V(v)=\frac{d}{dv}[F_V(v)]$$
        $$f_V(v)=2c(v-2), 2\leq v<10$$

        We then know:

        $$\int_2^{10} 2c(v-2)\,dv=1$$

        We can solve to get:

        $$2c\left[ \frac{v^2}{2}-2v\Big|_2^{10} \right]=1$$
        $$2c\left[ (50-2)-(20-4) \right]=1$$
        $$64c=1$$

        Which finally gets us:

        $$\boxed{c=64}$$

      \item We can then find the probability that the response time is greater than $5[\si{\milli\second}]$ as:

        $$P(v>5)=1-F_V(v)$$
        $$P(v>5)=1-\frac{1}{64}(5-2)^2$$
        $$\boxed{P(v>5)=\frac{55}{64}}$$

      \item We can then find the response time probability for between 5 and 8 milliseconds:

        $$P(5\leq v<8)=F_V(8)-F_V(5)$$
        $$P(5\leq v<8)=\frac{1}{64}[(8-2)^2-(5-2)^2]$$
        $$\boxed{P(5\leq v<8)=\frac{27}{64}}$$

      \item We can find this to be:

        $$P(v>7|5\leq v\leq8)=\frac{P(7<v\leq 8)}{P(5\leq v\leq 8)}$$

        We find the probability of the numerator:

        $$P(7<v\leq 8)=F_V(8)-F_V(7)$$
        $$P(7<v\leq 8)=\frac{1}{64}[(8-2)^2-(7-2)^2]$$
        $$P(7<v\leq 8)=\frac{11}{64}$$

        This gives us:

        $$P(v>7|5\leq v\leq8)=\frac{11/64}{27/64}$$
        $$\boxed{P(v>7|5\leq v\leq8)=\frac{11}{27}}$$

      \item To find the applicable value, we may write:

        $$1-F_V(a)=.36$$

        We expand this to write:

        $$1-\frac{1}{64}(a-2)^2=.36$$

        We then solve:

        $$a=\sqrt{64(.64)}+2$$
        $$a=\pm6.4+2$$

        Since the time has to be positive, we find:

        $$\boxed{a=8.4[\si{\milli\second}]}$$

    \end{enumerate}

  \item

    \begin{enumerate}

      \item To be a valid PDF, we know:

        $$\int_{-\infty}^{\infty} ae^{-.2|x|}\,dx=1$$

        We expand:

        $$\int_{-\infty}^{0} ae^{.2x}\,dx+\int_0^{\infty} ae^{-.2x}\,dx=1$$

        This gives us:

        $$ \frac{ae^{.2x}}{.2}\Big|_{-\infty}^0-\frac{ae^{-.2x}}{.2}\Big|_{0}^{\infty}=1$$

        We continue to solve:

        $$(5a-0)-(0-5a)=1$$
        $$10a=1$$
        $$\boxed{a=.1}$$

      \item We know that the expectation value can be expressed as:

        $$E[x]=\int_{-\infty}^{\infty} xf_X(x)\,dx$$

        This gives us:

        $$E[x]=\frac{1}{10}\left[\int_{-\infty}^{0} xe^{.2x}\,dx+\int_0^{\infty}xe^{-.2x}\,dx\right]$$
        $$\boxed{E[x]=0[\si{\micro\volt}]}$$

        We may observe that, due to symmetry about the $x$ axis, the expected value is zero.

      \item Once again, we break up the function to find the CDF in two regions. We begin with the first region:

        $$F_X(x<0)=\int_{-\infty}^x .1e^{.2x}\,dx$$

        This gives us:

        $$F_X(x<0)=.5e^{.2x}\Big|_{-\infty}^x$$
        $$F_X(x<0)=.5e^{.2x}$$

        We then find the second region:

        $$F_X(x\geq0)=\int_{-\infty}^0 .1e^{.2x}\,dx+\int_{0}^{x} .1e^{-.2x}\,dx$$

        This gives us:

        $$F_X(x\geq0)=.5e^{.2x}+\left[-.5e^{-.2x}\right]\Big|_0^{x}$$
        $$F_X(x\geq0)=1-.5e^{-.2x}$$

        Finally, we may express this as:

        $$\boxed{F_X(x)=\left\{\begin{array}{ll} .5e^{.2x},& x<0\\ 1-.5e^{-.2x},&x\geq 0\end{array}}$$

      \item We can calculate the former as:

        $$P[X>0]=1-P[X\leq 0]$$
        $$P[X>0]=1-\int_{-\infty}^0 f_X(x)\,dx$$

        This gives us:

        $$P[X>0]=1-\left[ .5e^{.2x} \right]\Big|_{-\infty}^{0}$$
        $$\boxed{P[X>0]=.5}$$

        Similarly, we may find:

        $$P[X>2]=1-P[X\leq 2]$$

        This gives us:

        $$P[X>2]=1-P[X>0]-\int_0^2 .1e^{-.2x}\,dx$$

        We continue to evaluate:

        $$P[X>2]=1-.5-\left[-.5e^{-.2x}\right]\Big_0^2$$
        $$P[X>2]=1-.5-[-.5e^{-.4}-(-.5)]$$
        $$P[X>2]=.5e^{-.4}$$
        $$\boxed{P[X>2]=.3352}$$

      \item Given that the PDF is symmetrical, we may simply write:

        $$P[|X|>2]=2P[X>2]$$
        $$P[|X|>2]=2(.3352)$$
        $$\boxed{P[|X|>2]=.6704}$$

    \end{enumerate}

  \item

    \begin{enumerate}

      \item Given that the PDF can be found as the differential of the CDF, we get:

        $$f_X(x)=\frac{d}{dx}\left[ F_X(x) \right]$$
        $$f_X(x)=\frac{1}{16}\frac{d}{dx}\left[ (x-2)^2 \right],\quad 2\leq x<6$$

        This gives us:

        $$\boxed{f_X(x)=\frac{x-2}{8},\quad 2\leq x<6}$$

      \item We can find the expected value using the formula:

        $$E[X]=\int_{-\infty}^{\infty} xf_X(x)\,dx$$

        This gives us:

        $$E[X]=\int_{-\infty}^{\infty} x\left( \frac{x-2}{8} \right)\,dx$$
        $$E[X]=\frac{1}{8}\int_{2}^{6} \left( x^2-2x \right)\,dx$$

        We evaluate to find:

        $$E[X]=\frac{1}{8}\left[\frac{x^3}{3}-x^2\right]\Big|_2^6$$
        $$E[X]=\frac{1}{8}\left[(72-36)-\left( \frac{8}{3}-4 \right)\right]$$
        $$\boxed{E[X]=\frac{112}{24}=4.6\bar{6}[\si{\milli\second}]}$$

        Using this mean value, we can find the variance as:

        $$\text{Var}[X]=\int_2^6 (x-4.6\bar{6})^2f_X(x)\,dx$$

        We evaluate to get:

        $$\boxed{\text{Var}[X]=.88\bar{8}[\si{\milli\second\squared}]}$$

      \item Using the PDF, we may find $P[X>4]=1-P[X\leq 4]$. This gives us:

        $$P[X>4]=1-\int_2^4 f_X(x)\,dx$$
        $$P[X>4]=1-\int_2^4 \frac{x-2}{8}\,dx$$

        We evaluate to get:

        $$P[X>4]=1-\frac{1}{16}(x^2-2x)\Big|_2^4$$
        $$P[X>4]=1-\frac{1}{16}[(16-8)-(4-4)]$$
        $$\boxed{P[X>4]=\frac{1}{2}=.5}$$

        From here, we may determine the conditional probability of:

        $$P[X>5|X>4]=\frac{P[X>5]}{P[X>4]}$$

        From here, we find:

        $$P[X>5]=1-\frac{1}{16}(x^2-2x)\Big|_2^5$$
        $$P[X>5]=1-\frac{1}{16}[(25-10)-(4-4)]$$
        $$\boxed{P[X>5]=\frac{1}{16}=.0625}$$

        We then find:

        $$P[X>5|X>4]=\frac{.0625}{.5}$$
        $$\boxed{P[X>5|X>4]=.125}$$

      \item We know that the PDF will scale such that:

        $$f_{X|A}(x)=\frac{f_X(x)}{P[A]}$$

        Thus, we can find:

        $$P[A]=1-\frac{1}{16}(x^2-2x)\Big|_2^3$$
        $$P[A]=1-\frac{1}{16}[(9-6)-(4-4)]\Big|_2^3$$
        $$P[A]=\frac{13}{16}=.8125$$

        Thus, we scale the initial pdf to get:

        $$f_{X|A}(x)=\frac{16}{13}\left[ \frac{x-2}{8} \right]$$
        $$\boxed{f_{X|A}(x)=\frac{2x-4}{13},\quad 3\leq x<6}$$

    \end{enumerate}

  \item

    \begin{enumerate}

      \item We know that the PDF is continuous and equivalent over the whole interval, and, thus, we may write:

        $$f_X(x)=\frac{1}{a-(-a)},\quad -a\leq x\leq a$$
        $$\boxed{f_X(x)=\frac{1}{2a},\quad -a\leq x\leq a}$$

      \item 

        Given that we know the mean will be zero, we may write:

        $$\text{Var}[X]=\int_{-a}^a (x-0)^2\frac{1}{2a}\,dx$$
        $$\text{Var}[X]=\int_{-a}^a \frac{x^2}{2a}\,dx$$

        We then evaluate:

        $$\text{Var}[X]=\frac{x^3}{6a}\Big_{-a}^a$$
        $$\text{Var}[X]=\left[\frac{a^3}{6a}-\left( -\frac{a^3}{6a} \right)\right]$$
        $$\text{Var}[X]=\left[\frac{a^2}{6}+\frac{a^2}{6} \right]$$
        $$\boxed{\text{Var}[X]=\frac{a^2}{3}}$$

      \item We can write:

        $$P\left[X>\frac{a}{2}|X>0\right]=\frac{P\left[X>\frac{a}{2}\right]}{P[X>0]}$$

        We begin by finding:

        $$P[X>0]=1-\frac{0-(-a)}{2a}$$
        $$P[X>0]=\frac{1}{2}$$

        We then find:

        $$P\left[ X>\frac{a}{2} \right]=1-\frac{\frac{a}{2}-(-a)}{a-(-a)}$$
        $$P\left[ X>\frac{a}{2} \right]=\frac{1}{4}$$

        We then divide to get:

        $$P\left[X>\frac{a}{2}|X>0\right]=\frac{.25}{.5}$$
        $$\boxed{P\left[X>\frac{a}{2}|X>0\right]=\frac{1}{2}}$$

      \item Given the uniform probability, we may simply double our value of $P\left[ X>\frac{a}{2} \right]$ to find:

        $$P\left[ |X|>\frac{a}{2} \right]=2P\left[ X>\frac{a}{2} \right]$$

        This then gives us the probability of a fault trigger as:

        $$\boxed{P\left[ |X|>\frac{a}{2} \right]=\frac{1}{2}}$$

      \item We begin by setting the equation appropriately to get:

        $$2X^2>\frac{a^2}{3}$$

        Solving this gives us:

        $$|X|>\frac{a}{\sqrt{6}}$$

        Thus, we want to find:

        $$P\left[|X|>\frac{a}{\sqrt{6}}\right]=2P\left[ X>\frac{a}{\sqrt{6}} \right]$$

        This gives us:

        $$P\left[|X|>\frac{a}{\sqrt{6}}\right]=2\left[ 1-\frac{\frac{a}{\sqrt{6}}-(-a)}{2a} \right]$$
        $$\boxed{P\left[|X|>\frac{a}{\sqrt{6}}\right]=.5918}$$

    \end{enumerate}

    \setcounter{enumi}{5}

  \item

    \begin{enumerate}

      \item For an exponential random variable, we know:

        $$\lambda=\frac{1}{\mu}$$

        Which gives us:

        $$\lambda=\frac{1}{2}$$

        Thus, we may write the PDF as:

        $$\boxed{f_X(x)=.5e^{-.5x},\quad x>0}$$

      \item First, we want to find the probability of the request taking less than two seconds:

        $$P[X<2]=\int_0^2 .5e^{-.5x}\,dx$$
        $$P[X<2]=-e^{-.5x}\Big|_0^2$$
        $$P[X<2]=[-e^{-1}-(-1)]$$
        $$\boxed{P[X<2]=.6321}$$

        From here, since each request is independent, we simply find:

        $$P[X_3<2]=(P[X<2])^3$$
        $$P[X_3<2]=(.6321)^3$$
        $$\boxed{P[X_3<2]=.2526}$$

      \item 

        \begin{enumerate}

          \item Although the $\lambda$ value stays the same as the above, we define $n$ as the quantity of requests, or $3$, such that:

            $$\boxed{Y=\text{Erlang}(3,.5)}$$

          \item Using the Erlang distribution, we write:

            $$\boxed{F_Y(y)=.0625y^2e^{-.5y}}$$

          \item We can find the mean value as:

            $$E[Y]=\int_0^{\infty} .0625y^2e^{-.5y}\,dy$$

            We solve to get:

            $$\boxed{E[Y]=6}$$

            Then, we find the variance:

            $$\text{Var}[Y]=\int_0^{\infty} .0625(y-6)^2y^2e^{-.5y}\,dy$$

            Solving gives us:

            $$\boxed{\text{Var}[Y]=12}$$

          \item We can find the probability as:

            $$P[Y<5]=\int_0^5 .0625y^2e^{-.5y}\,dy$$

            Solving gives us:

            $$\boxed{P[Y<5]=.45618}$$

        \end{enumerate}

    \end{enumerate}

    \setcounter{enumi}{7}

  \item

    \begin{enumerate}

      \item Given that we want 95\% of all readings to be within $1^o$ of $\mu$, we can use the inverse normal function to find a z-score of:

        $$\frac{x-\mu}{\sigma}=\pm 1.96$$

        Furthermore, we know that:

        $$x-\mu=\pm 1$$

        We combine the two to get:

        $$\pm 1.96\sigma=\pm 1$$
        $$\boxed{\sigma=.5102}$$

      \item Using our z-score formula, we may write:

        $$z=\frac{25-\mu}{1}=25-\mu$$

        We know that:

        $$P[Y\leq 25^o]=P[Z\leq 25-\mu]$$

        Given the probability for $Y\leq25^o$, we can use the inverse normal function to write:

        $$z=1.4985$$

        This gives us:

        $$25-\mu=1.4985$$
        $$\boxed{\mu=23.501^o}$$

      \item Similarly, we may find:

        $$z=-1.9954$$

        From this, we write:

        $$-\mu=-1.9954$$
        $$\boxed{\mu=1.9954^o}$$

      \item If the probability above or below a certain point is equal to 1/2, then we are at the mean. Therefore, we may write:

        $$\boxed{\mu=25^o}$$

    \end{enumerate}

  \item

    \begin{enumerate}

      \item We want to find a value for which $90\%$ are less than or equal to, in terms of storage. We may begin by taking the inverse of the normal to find the $z$ score:

        $$\text{invnorm}(.9)=1.2816$$

        Now that we know the $z$ score, we may write:

        $$\frac{X-500}{100}=1.2816$$

        Solving, we find:

        $$X=100(1.2816)+500$$

        Given that we want 90\% of users to be 30 GB below the set threshold, we may find: 

        $$\boxed{W_{Th}=658.16[\si{GB}]}$$

      \item We can calculate this by transferring to a $z$ score:

        $$z=\frac{600-500}{100}$$
        $$z=1$$

        And then taking the normal function to get:

        $$\%_{\text{users}}=1-.8413$$
        $$\boxed{\%_{\text{users}}=15.87\%}$$

      \item Given that we can write the initial PDF as:

        $$f_W(w)=\text{NormalPDF}(500,100)$$

        We can write the conditional probability as:

        $$f_{W|A}(w)=\frac{P[A]}{\text{NormalPDF}(500,100)}$$

        As such, we use our probability from above to write:

        $$f_{W|A}(w)=\left[\frac{6.303}{100\sqrt{2\pi}}e^{-\frac{(w-500)^2}{20000}}\right]^{-1}$$

        This can be simplified to:

        $$\boxed{f_{W|A}(w)=39.769e^{\frac{(w-500)^2}{20000}}}$$

      \item We can shift the PDF such that:

        $$E[(W-600)|A]=E[W|W>600]-600$$

        For a normal distribution, this gives us:

        $$E[W|W>600]=\mu+\sigma\left[ \frac{\Phi(z)}{1-\Phi(z)} \right]$$

        We use the previously-obtained $z$-score to get:

        $$E[W|W>600]=500+100\left[ \frac{\phi(1)}{1-\Phi(1)} \right]$$

        We can find:

        $$\phi(1)=\frac{1}{\sqrt{2\pi}}e^{-.5}$$
        $$\phi(1)=.242$$

        We know from earlier that:

        $$1-\Phi(1)=.1587$$

        Thus, we get:

        $$E[W|W>600]=500+100\left[ \frac{.242}{.1587} \right]$$
        $$E[W|W>600]=652.47[\si{GB}]$$

        From here, we can obtain the value we want:

        $$E[(W-600)|A]=652.47-600$$
        $$\boxed{E[(W-600)|A]=52.47[\si{GB}]}$$

        The expected revenue per user is then:

        $$R=52.47\cdot.05$$
        $$R=2.6235\$\left[ \si{Dollars\over Users} \right]$$

        With 10,000 users, we get:

        $$\boxed{R_{10k}=26,235\$\left[ \si{Dollars} \right]}$$

    \end{enumerate}

  \item

    \begin{enumerate}

      \item The event $X\geq t$ is the same as $A_1\cap A_2\cap A_3$

      \item Given that the events are independent, along with the statement from part (a), we may write that an individual event can be expressed as:

        $$P[A_i]=e^{-.01t}$$

        And therefore $P[X\geq t]$ can be expressed as:

        $$P[X\geq t]=\prod_{i=1}^3 P[A_i]$$
        $$P[X\geq t]=\left( e^{-.01t} \right)^3$$
        $$\boxed{P[X\geq t]=e^{-.03t}}$$

        Since the above is the probability that $X$ is greater than or equal to a certain time, we can find $F_X(t)=P[X< t]$ as:

        $$\boxed{F_X(t)=1-e^{-.03t}}$$

        We can then differentiate to find:

        $$\boxed{f_X(t)=.03e^{-.03t}}$$

        We may observe that this is an exponential (Laplace) random distribution.

      \item Per our distribution rules we know:

        $$\lambda=.03$$

        And also:

        $$E[X]=\lambda^{-1}$$

        This gives us:

        $$\boxed{E[X]=33.33\bar{3}}$$

        Furthermore, we know:

        $$\text{Var}[X]=\frac{1}{\lambda^2}$$

        This gives us:

        $$\boxed{\text{Var}[X]=1111.11}$$

    \end{enumerate}

  \item

    For the first plan, we may simply write:

    $$\boxed{Y_A=.05X}$$

    For the second plan, we have a two-step system, for which we get:

    $$\boxed{Y_B=\left\{\begin{array}{ll} .2,& X\leq 10\\ .05X-.3,& X>10\end{array}}$$

    Given the scaling factor, we may write the expected value of $Y_A$ as:

    $$E[Y_A]=E[.05X]$$
    $$E[Y_A]=.05E[X]$$
    $$\boxed{E[Y_A]=.05\mu}$$

    For the expectation value of the second plan, the process is a bit more complex. Given that the expectation value is an exponential random variable, we can begin by writing the PDF:

    $$f_X(x)=\frac{1}{\mu}e^{-\frac{x}{\mu}}$$

    From here, we can write out the expectation values as:

    $$Y_B=.2P[X\leq10]+E[.05X-.3|X>10]P[X>10]$$

    From the PDF, we may integrate to get the PDF as:

    $$F_X(X\leq x)=1-e^{-\frac{x}{\mu}}$$

    As such, we may find:

    $$P[X\leq 10]=1-e^{-\frac{10}{\mu}}\quad\text{ and }\quad P[X>10]=e^{-\frac{10}{\mu}}$$

    We plug this into our expectation value formula above to get:

    $$Y_B=.2\left( 1-e^{-\frac{10}{\mu}} \right)+(.2+.05\mu)e^{-\frac{10}{\mu}}$$

    We simplify to get:

    $$\boxed{Y_B=.2+.05\mu e^{-\frac{10}{\mu}}}$$

    We then use the above formulas at $\mu=5,8[\si{GB}]$ to get:

    $$\$_{A,5}=.05(5)=.25[\si{Dollars}]$$
    $$\$_{A,8}=.05(8)=.4[\si{Dollars}]$$
    $$\$_{B,5}=.2+.05(5)e^{-2}=.2338[\si{Dollars}]$$
    $$\$_{B,8}=.2+.05(8)e^{-(10/8)}=.3146[\si{Dollars}]$$

    \underline{As such, Plan B is better in both cases}

\end{enumerate}

\end{document}

