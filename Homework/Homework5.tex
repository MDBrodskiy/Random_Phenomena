%%%%%%%%%%%%%%%%%%%%%%%%%%%%%%%%%%%%%%%%%%%%%%%%%%%%%%%%%%%%%%%%%%%%%%%%%%%%%%%%%%%%%%%%%%%%%%%%%%%%%%%%%%%%%%%%%%%%%%%%%%%%%%%%%%%%%%%%%%%%%%%%%%%%%%%%%%%%%%%%%%%
% Written By Michael Brodskiy
% Class: Analysis of Random Phenomena
% Professor: I. Salama
%%%%%%%%%%%%%%%%%%%%%%%%%%%%%%%%%%%%%%%%%%%%%%%%%%%%%%%%%%%%%%%%%%%%%%%%%%%%%%%%%%%%%%%%%%%%%%%%%%%%%%%%%%%%%%%%%%%%%%%%%%%%%%%%%%%%%%%%%%%%%%%%%%%%%%%%%%%%%%%%%%%

\include{Includes.tex}

\title{Homework 5}
\date{\today}
\author{Michael Brodskiy\\ \small Professor: I. Salama}

\begin{document}

\maketitle

\begin{enumerate}

  \item

    \begin{enumerate}

      \item To be a valid CDF, we know that the terms continuously build until they sum to 1. In this case, all of the terms become 1 at $v=10$. Thus, we can take the sum as:

        $$c\sum_{n=2}^{9} (v-2)^2=1$$

        Iterating, we find:

        $$c(0 + 1 + 4 + 9 + 16 + 25 + 36 + 49)=1$$

        This gives us:

        $$\boxed{c=\frac{1}{140}}$$

      \item 

      \item 

      \item 

      \item 

    \end{enumerate}

  \item

    \begin{enumerate}

      \item 

      \item 

      \item 

      \item 

      \item 

    \end{enumerate}

  \item

    \begin{enumerate}

      \item 

      \item 

      \item 

      \item 

      \item 

    \end{enumerate}

  \item

    \begin{enumerate}

      \item 

      \item 

      \item 

      \item 

      \item 

    \end{enumerate}

    \setcounter{enumi}{5}

  \item

    \begin{enumerate}

      \item 

      \item 

      \item 

        \begin{enumerate}

          \item 

          \item 

          \item 

          \item 

        \end{enumerate}

    \end{enumerate}

    \setcounter{enumi}{7}

  \item

    \begin{enumerate}

      \item 

      \item 

      \item 

      \item 

    \end{enumerate}

  \item

    \begin{enumerate}

      \item 

      \item 

      \item 

      \item 

    \end{enumerate}

  \item

    \begin{enumerate}

      \item 

      \item 

      \item 

    \end{enumerate}

  \item

\end{enumerate}

\end{document}

