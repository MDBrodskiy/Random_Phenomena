%%%%%%%%%%%%%%%%%%%%%%%%%%%%%%%%%%%%%%%%%%%%%%%%%%%%%%%%%%%%%%%%%%%%%%%%%%%%%%%%%%%%%%%%%%%%%%%%%%%%%%%%%%%%%%%%%%%%%%%%%%%%%%%%%%%%%%%%%%%%%%%%%%%%%%%%%%%%%%%%%%%
% Written By Michael Brodskiy
% Class: Analysis of Random Phenomena
% Professor: I. Salama
%%%%%%%%%%%%%%%%%%%%%%%%%%%%%%%%%%%%%%%%%%%%%%%%%%%%%%%%%%%%%%%%%%%%%%%%%%%%%%%%%%%%%%%%%%%%%%%%%%%%%%%%%%%%%%%%%%%%%%%%%%%%%%%%%%%%%%%%%%%%%%%%%%%%%%%%%%%%%%%%%%%

\documentclass[12pt]{article} 
\usepackage{alphalph}
\usepackage[utf8]{inputenc}
\usepackage[russian,english]{babel}
\usepackage{titling}
\usepackage{float}
\usepackage{amsmath}
\usepackage{graphicx}
\usepackage{enumitem}
\usepackage{amssymb}
\usepackage[super]{nth}
\usepackage{everysel}
\usepackage{ragged2e}
\usepackage{geometry}
\usepackage{multicol}
\usepackage{fancyhdr}
\usepackage{cancel}
\usepackage{siunitx}
\usepackage{physics}
\usepackage{tikz}
\usepackage{mathdots}
\usepackage{yhmath}
\usepackage{cancel}
\usepackage{color}
\usepackage{xcolor}
\usepackage{colortbl}
\usepackage{array}
\usepackage{multirow}
\usepackage{gensymb}
\usepackage{tabularx}
\usepackage{extarrows}
\usepackage{booktabs}
\usepackage{lastpage}
\usetikzlibrary{fadings}
\usetikzlibrary{patterns}
\usetikzlibrary{shadows.blur}
\usetikzlibrary{shapes}

\geometry{top=1.0in,bottom=1.0in,left=1.0in,right=1.0in}
\newcommand{\subtitle}[1]{%
  \posttitle{%
    \par\end{center}
    \begin{center}\large#1\end{center}
    \vskip0.5em}%

}
\usepackage{hyperref}
\hypersetup{
colorlinks=true,
linkcolor=blue,
filecolor=magenta,      
urlcolor=blue,
citecolor=blue,
}


\title{Homework 5}
\date{\today}
\author{Michael Brodskiy\\ \small Professor: I. Salama}

\begin{document}

\maketitle

\begin{enumerate}

  \item

    \begin{enumerate}

      \item To be a valid CDF, we know that the terms continuously build until they sum to 1. In this case, all of the terms become 1 at $v=10$. Thus, we can differentiate to find the PDF:

        $$f_V(v)=\frac{d}{dv}[F_V(v)]$$
        $$f_V(v)=2c(v-2), 2\leq v<10$$

        We then know:

        $$\int_2^{10} 2c(v-2)\,dv=1$$

        We can solve to get:

        $$2c\left[ \frac{v^2}{2}-2v\Big|_2^{10} \right]=1$$
        $$2c\left[ (50-2)-(20-4) \right]=1$$
        $$64c=1$$

        Which finally gets us:

        $$\boxed{c=64}$$

      \item We can then find the probability that the response time is greater than $5[\si{\milli\second}]$ as:

        $$P(v>5)=1-F_V(v)$$
        $$P(v>5)=1-\frac{1}{64}(5-2)^2$$
        $$\boxed{P(v>5)=\frac{55}{64}}$$

      \item We can then find the response time probability for between 5 and 8 milliseconds:

        $$P(5\leq v<8)=F_V(8)-F_V(5)$$
        $$P(5\leq v<8)=\frac{1}{64}[(8-2)^2-(5-2)^2]$$
        $$\boxed{P(5\leq v<8)=\frac{27}{64}}$$

      \item We can find this to be:

        $$P(v>7|5\leq v\leq8)=\frac{P(7<v\leq 8)}{P(5\leq v\leq 8)}$$

        We find the probability of the numerator:

        $$P(7<v\leq 8)=F_V(8)-F_V(7)$$
        $$P(7<v\leq 8)=\frac{1}{64}[(8-2)^2-(7-2)^2]$$
        $$P(7<v\leq 8)=\frac{11}{64}$$

        This gives us:

        $$P(v>7|5\leq v\leq8)=\frac{11/64}{27/64}$$
        $$\boxed{P(v>7|5\leq v\leq8)=\frac{11}{27}}$$

      \item To find the applicable value, we may write:

        $$1-F_V(a)=.36$$

        We expand this to write:

        $$1-\frac{1}{64}(a-2)^2=.36$$

        We then solve:

        $$a=\sqrt{64(.64)}+2$$
        $$a=\pm6.4+2$$

        Since the time has to be positive, we find:

        $$\boxed{a=8.4[\si{\milli\second}]}$$

    \end{enumerate}

  \item

    \begin{enumerate}

      \item To be a valid PDF, we know:

        $$\int_{-\infty}^{\infty} ae^{-.2|x|}\,dx=1$$

        We expand:

        $$\int_{-\infty}^{0} ae^{.2x}\,dx+\int_0^{\infty} ae^{-.2x}\,dx=1$$

        This gives us:

        $$ \frac{ae^{.2x}}{.2}\Big|_{-\infty}^0-\frac{ae^{-.2x}}{.2}\Big|_{0}^{\infty}=1$$

        We continue to solve:

        $$(5a-0)-(0-5a)=1$$
        $$10a=1$$
        $$\boxed{a=.1}$$

      \item We know that the expectation value can be expressed as:

        $$E[x]=\frac{1}{\lambda}$$

        Where $\lambda$ is the coefficient of the exponent. Thus, we get:

        $$E[x]=\frac{1}{.2}$$
        $$\boxed{E[x]=5}$$

      \item 

      \item 

      \item 

    \end{enumerate}

  \item

    \begin{enumerate}

      \item 

      \item 

      \item 

      \item 

      \item 

    \end{enumerate}

  \item

    \begin{enumerate}

      \item 

      \item 

      \item 

      \item 

      \item 

    \end{enumerate}

    \setcounter{enumi}{5}

  \item

    \begin{enumerate}

      \item 

      \item 

      \item 

        \begin{enumerate}

          \item 

          \item 

          \item 

          \item 

        \end{enumerate}

    \end{enumerate}

    \setcounter{enumi}{7}

  \item

    \begin{enumerate}

      \item 

      \item 

      \item 

      \item 

    \end{enumerate}

  \item

    \begin{enumerate}

      \item 

      \item 

      \item 

      \item 

    \end{enumerate}

  \item

    \begin{enumerate}

      \item 

      \item 

      \item 

    \end{enumerate}

  \item

\end{enumerate}

\end{document}

