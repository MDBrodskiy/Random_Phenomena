%%%%%%%%%%%%%%%%%%%%%%%%%%%%%%%%%%%%%%%%%%%%%%%%%%%%%%%%%%%%%%%%%%%%%%%%%%%%%%%%%%%%%%%%%%%%%%%%%%%%%%%%%%%%%%%%%%%%%%%%%%%%%%%%%%%%%%%%%%%%%%%%%%%%%%%%%%%%%%%%%%%
% Written By Michael Brodskiy
% Class: Analysis of Random Phenomena
% Professor: I. Salama
%%%%%%%%%%%%%%%%%%%%%%%%%%%%%%%%%%%%%%%%%%%%%%%%%%%%%%%%%%%%%%%%%%%%%%%%%%%%%%%%%%%%%%%%%%%%%%%%%%%%%%%%%%%%%%%%%%%%%%%%%%%%%%%%%%%%%%%%%%%%%%%%%%%%%%%%%%%%%%%%%%%

\documentclass[12pt]{article} 
\usepackage{alphalph}
\usepackage[utf8]{inputenc}
\usepackage[russian,english]{babel}
\usepackage{titling}
\usepackage{float}
\usepackage{amsmath}
\usepackage{graphicx}
\usepackage{enumitem}
\usepackage{amssymb}
\usepackage[super]{nth}
\usepackage{everysel}
\usepackage{ragged2e}
\usepackage{geometry}
\usepackage{multicol}
\usepackage{fancyhdr}
\usepackage{cancel}
\usepackage{siunitx}
\usepackage{physics}
\usepackage{tikz}
\usepackage{mathdots}
\usepackage{yhmath}
\usepackage{cancel}
\usepackage{color}
\usepackage{xcolor}
\usepackage{colortbl}
\usepackage{array}
\usepackage{multirow}
\usepackage{gensymb}
\usepackage{tabularx}
\usepackage{extarrows}
\usepackage{booktabs}
\usepackage{lastpage}
\usetikzlibrary{fadings}
\usetikzlibrary{patterns}
\usetikzlibrary{shadows.blur}
\usetikzlibrary{shapes}

\geometry{top=1.0in,bottom=1.0in,left=1.0in,right=1.0in}
\newcommand{\subtitle}[1]{%
  \posttitle{%
    \par\end{center}
    \begin{center}\large#1\end{center}
    \vskip0.5em}%

}
\usepackage{hyperref}
\hypersetup{
colorlinks=true,
linkcolor=blue,
filecolor=magenta,      
urlcolor=blue,
citecolor=blue,
}


\title{Homework 4}
\date{\today}
\author{Michael Brodskiy\\ \small Professor: I. Salama}

\begin{document}

\maketitle

\begin{enumerate}

  \item 

    \begin{enumerate}

      \item Given the Poisson distribution with given mean of $\alpha=.1$ interruptions per minute gives us a PMF of:

        $$\boxed{P_{\alpha}(n)=\frac{(.1)^ne^{-.1}}{n!}}$$

      \item The expectation value is simply given as the average value, such that:

        $$\boxed{E[X]=.1}$$

        Furthermore, we know that the same value represents the standard deviation. Since we know that the standard deviation is the square root of the variance we get:

        $$\sigma=\sqrt{.1}$$
        $$\boxed{\sigma=.3162}$$

      \item Given a 10 minute period, we may take $\alpha\to 10(.1)\to 1$. This gives us:

        $$P_{10}(n)=\frac{(1)^ne^{-1}}{n!}$$

        We want the probability of no events, so we may take $n\to 0$ to get:

        $$P_{10}(0)=\frac{(1)^0e^{-1}}{0!}$$
        $$\boxed{P_{10}(0)=.3679}$$

      \item Given a 20 minute period, we may take $\alpha\to 20(.1)\to 2$. This gives us:

        $$P_{20}(n)=\frac{(2)^ne^{-2}}{n!}$$

        We want the probability of two or more events, so we may take the complement of 2, 1 or no events. This gives us:

        $$P_{20}(>2)=1-\left(\frac{(2)^0e^{-2}}{0!}+\frac{(2)^1e^{-2}}{1!}+\frac{(2)^2e^{-2}}{2!}\right)$$

      \item 

      \item 

    \end{enumerate}

  \item 

    \begin{enumerate}

      \item 

      \item 

      \item 

    \end{enumerate}

  \item First and foremost, we see that this CDF is valid, since the probabilities add up to 100\%, or $1$.

    \begin{enumerate}

      \item We can find the first value as:

        $$P[Y<3] = F[2]$$
        $$\boxed{P[Y<3] =.25}$$

        We can then find:

        $$P[Y\leq 3]=F[3]$$
        $$\boxed{P[Y\leq3] =.5}$$

      \item We can see that:

        $$P[Y< 4]=P[Y\leq 3]=F[3]$$
        $$\boxed{P[Y< 4]=.5}$$

        From here, we may find:

        $$P[Y\geq 4]=1-P[Y<4]$$
        $$\boxed{P[Y\geq 4]=.5}$$

      \item Since there is no ``bump'' up at $y=2$, we may find:

        $$\boxed{P[Y=2]=F[2]-F[1]=0}$$

        Similarly, we may find:

        $$\boxed{P[1\leq Y<3]=F[2]-F[1]=0}$$

      \item We know that the PMF may be expressed as:

        $$PMF(Y)=F[Y]-F[Y-1]$$

        Using this, we construct:

        $$\boxed{PMF(Y)=\left\{ \begin{array}{ll} .25, & y=1\\.25, & y=3\\.5, & y=4\\ 0, & \text{otherwise} \end{array}}$$

    \end{enumerate}

  \item 

    \begin{enumerate}

      \item We may observe that this is a geometric distribution, which means that the expectation may be written as:

        $$E[K]=\frac{1}{p}=\frac{1}{.05}$$
        $$\boxed{E[K]=20}$$

        From here, we may find the variance:

        $$\text{Var}[K]=\frac{1-p}{p^2}=\frac{.95}{.05^2}$$
        $$\boxed{\text{Var}[K]=380}$$

        And finally, we use this to find the standard deviation:

        $$\sigma_K=\sqrt{\text{Var}[K]}=\sqrt{380}$$
        $$\boxed{\sigma_K=19.494}$$

      \item We may write the CDF using a sum and the formula for a geometric distribution:

        $$\boxed{\text{CDF}_K[n]=\sum_{n=1}^{n_K}(.05)(.95)^{n-1}}$$

        Where $n$ represents the attempt number and $n_K$ is the number of attempts until an error is encountered.

      \item Let us find the probability of observing the expectation value:

        $$P[K=E[K]]=(.05)(.95)^{19}$$

        This gives us:

        $$\boxed{P[K=20]=.018868}$$

      \item We can find the probability that more attempts than expected are taken as:

        $$P[K>20]=1-\sum_{n=1}^{20}(.05)(.95)^{n-1}$$

        This gives us:

        $$\boxed{P[K>20]=.3585}$$

      \item On the other hand, we may find the probability that less attempts than expected are needed:

        $$P[K<20]=1-P[K>20]-P[K=20]$$
        $$P[K<20]=1-.3585-.018868$$
        $$\boxed{P[K<20]=.6226}$$

      \item Using the standard deviation we obtained above, we may round the difference (or sum) of it and the expectation value to get::

        $$P[(20-19.494)\leq K \leq (20+19.494)]=P[1\leq K\leq 39]$$

        From here, we may use our formula to write:

        $$P[1\leq K\leq 39]=\sum_{1}^{39} (.05)(.95)^{n-1}$$

        Plugging this in to a solver, we get:

        $$\boxed{P[1\leq K\leq 39]=.8647}$$

      \item By Bayes' rule, we know that, since the two events are independent, the probability that there is success $\to$ success $\to$ failure is simply the probability given by two successes and a failure. We can then find the probability that the next three events are two successes and one failure to get:

        $$\boxed{P[2,1]=(.95)^2(.05)=.045125}$$

      \item Since we know that an error is expected to occur after twenty attempts, we would expected for there to be another error twenty attempts after an initial one. As such, with a 2-second request time, we may simply write the time between errors as:

        $$\boxed{t=20(2)=40[\si{\second}]}$$

    \end{enumerate}

  \item 

    \begin{enumerate}

      \item 

      \item 

      \item 

    \end{enumerate}

  \item 

    \begin{enumerate}

      \item 

      \item 

      \item 

    \end{enumerate}

  \item 

    \begin{enumerate}

      \item 

      \item 

      \item 

      \item 

      \item 

      \item 

    \end{enumerate}

  \item 

    \setcounter{enumi}{9}

  \item 

    \begin{enumerate}

      \item 

      \item 

    \end{enumerate}

\end{enumerate}

\end{document}

