%%%%%%%%%%%%%%%%%%%%%%%%%%%%%%%%%%%%%%%%%%%%%%%%%%%%%%%%%%%%%%%%%%%%%%%%%%%%%%%%%%%%%%%%%%%%%%%%%%%%%%%%%%%%%%%%%%%%%%%%%%%%%%%%%%%%%%%%%%%%%%%%%%%%%%%%%%%%%%%%%%%
% Written By Michael Brodskiy
% Class: Analysis of Random Phenomena
% Professor: I. Salama
%%%%%%%%%%%%%%%%%%%%%%%%%%%%%%%%%%%%%%%%%%%%%%%%%%%%%%%%%%%%%%%%%%%%%%%%%%%%%%%%%%%%%%%%%%%%%%%%%%%%%%%%%%%%%%%%%%%%%%%%%%%%%%%%%%%%%%%%%%%%%%%%%%%%%%%%%%%%%%%%%%%

\include{Includes.tex}

\title{Homework 4}
\date{\today}
\author{Michael Brodskiy\\ \small Professor: I. Salama}

\begin{document}

\maketitle

\begin{enumerate}

  \item 

    \begin{enumerate}

      \item Given the Poisson distribution with given time-average of $\lambda=.1$ interruptions per minute gives us a PMF of:

        $$\boxed{P_{\lambda t}(n)=\frac{(.1t)^ne^{-.1t}}{n!}}$$

        Where $t$ represents a given time interval.

      \item The expectation value is simply given as the average value, such that:

        $$\boxed{E[X]=.1}$$

        Furthermore, we know that the same value represents the variance. Since we know that the standard deviation is the square root of the variance we get:

        $$\sigma=\sqrt{.1}$$
        $$\boxed{\sigma=.3162}$$

      \item Given a 10 minute period, we may take $\alpha\to 10(.1)\to 1$. This gives us:

        $$P_{10}(n)=\frac{(1)^ne^{-1}}{n!}$$

        We want the probability of no events, so we may take $n\to 0$ to get:

        $$P_{10}(0)=\frac{(1)^0e^{-1}}{0!}$$
        $$\boxed{P_{10}(0)=.3679}$$

      \item Given a 20 minute period, we may take $\alpha\to 20(.1)\to 2$. This gives us:

        $$P_{20}(n)=\frac{(2)^ne^{-2}}{n!}$$

        We want the probability of more than two events, so we may take the complement of 2, 1 or no events. This gives us:

        $$P_{20}(>2)=1-\left(\frac{(2)^0e^{-2}}{0!}+\frac{(2)^1e^{-2}}{1!}+\frac{(2)^2e^{-2}}{2!}\right)$$

        And finally:

        $$\boxed{P_{20}(>2)=.3233}$$

      \item Since the two intervals are independent, the probabilities simply multiply. This gives us:

        $$P_{2\times10}(2)=[P_{10}(2)]^2=\left[\frac{(1)^2e^{-1}}{2!}\right]^2$$

        And finally:

        $$\boxed{P_{2\times10}(2)=.033834}$$

      \item We want to calculate:

        $$\left[ 1-e^{-.1t} \right]=.9$$

        We can solve this for $t$:

        $$e^{-.1t}=.1$$
        $$-.1t=-2.3026$$
        $$\boxed{t=23.026\left[ \si{minutes} \right]}$$

    \end{enumerate}

  \item 

    \begin{enumerate}

      \item We may express the PMF as a binomial distribution, offset by $3$ since we are trying to find the third device. This gives us:

        $$\boxed{P_X[k]=\left( \begin{matrix} k-1\\2\end{matrix} \right)(.7)^3(.3)^{k-3}}$$

      \item We want to find the probability that the third success occurs on the 5th attempt, or $k=5$. Thus, we get:

        $$P_X[5]=\left( \begin{matrix} 4\\2\end{matrix} \right)(.7)^3(.3)^{2}$$
        $$\boxed{P_X[5]=.1852}$$

      \item This can be expressed as $k\geq 5$. To find this, we take the complement of $k=3,4$ to get:

        $$P_X[k\geq5]= 1-\sum_{3}^4\left( \begin{matrix}k-1\\2\end{matrix} \right)(.7)^3(.3)^{k-1}$$

        This gives us:

        $$\boxed{P_X[k\geq5]= .3483}$$

    \end{enumerate}

  \item First and foremost, we see that this CDF is valid, since the probabilities add up to 100\%, or $1$.

    \begin{enumerate}

      \item We can find the first value as:

        $$P[Y<3] = F[2]$$
        $$\boxed{P[Y<3] =.25}$$

        We can then find:

        $$P[Y\leq 3]=F[3]$$
        $$\boxed{P[Y\leq3] =.5}$$

      \item We can see that:

        $$P[Y< 4]=P[Y\leq 3]=F[3]$$
        $$\boxed{P[Y< 4]=.5}$$

        From here, we may find:

        $$P[Y\geq 4]=1-P[Y<4]$$
        $$\boxed{P[Y\geq 4]=.5}$$

      \item Since there is no ``bump'' up at $y=2$, we may find:

        $$\boxed{P[Y=2]=F[2]-F[1]=0}$$

        Similarly, we may find:

        $$\boxed{P[1\leq Y<3]=F[2]-F[1]=0}$$

      \item We know that the PMF may be expressed as:

        $$PMF(Y)=F[Y]-F[Y-1]$$

        Using this, we construct:

        $$\boxed{PMF(Y)=\left\{ \begin{array}{ll} .25, & y=1\\.25, & y=3\\.5, & y=4\\ 0, & \text{otherwise} \end{array}}$$

    \end{enumerate}

  \item 

    \begin{enumerate}

      \item We may observe that this is a geometric distribution, which means that the expectation may be written as:

        $$E[K]=\frac{1}{p}=\frac{1}{.05}$$
        $$\boxed{E[K]=20}$$

        From here, we may find the variance:

        $$\text{Var}[K]=\frac{1-p}{p^2}=\frac{.95}{.05^2}$$
        $$\boxed{\text{Var}[K]=380}$$

        And finally, we use this to find the standard deviation:

        $$\sigma_K=\sqrt{\text{Var}[K]}=\sqrt{380}$$
        $$\boxed{\sigma_K=19.494}$$

      \item We may write the CDF using a sum and the formula for a geometric distribution:

        $$\boxed{\text{CDF}_K[n]=\sum_{n=1}^{n_K}(.05)(.95)^{n-1}}$$

        Where $n$ represents the attempt number and $n_K$ is the number of attempts until an error is encountered.

      \item Let us find the probability of observing the expectation value:

        $$P[K=E[K]]=(.05)(.95)^{19}$$

        This gives us:

        $$\boxed{P[K=20]=.018868}$$

      \item We can find the probability that more attempts than expected are taken as:

        $$P[K>20]=1-\sum_{n=1}^{20}(.05)(.95)^{n-1}$$

        This gives us:

        $$\boxed{P[K>20]=.3585}$$

      \item On the other hand, we may find the probability that less attempts than expected are needed:

        $$P[K<20]=1-P[K>20]-P[K=20]$$
        $$P[K<20]=1-.3585-.018868$$
        $$\boxed{P[K<20]=.6226}$$

      \item Using the standard deviation we obtained above, we may round the difference (or sum) of it and the expectation value to get::

        $$P[(20-19.494)\leq K \leq (20+19.494)]=P[1\leq K\leq 39]$$

        From here, we may use our formula to write:

        $$P[1\leq K\leq 39]=\sum_{1}^{39} (.05)(.95)^{n-1}$$

        Plugging this in to a solver, we get:

        $$\boxed{P[1\leq K\leq 39]=.8647}$$

      \item By Bayes' rule, we know that, since the two events are independent, the probability that there is success $\to$ success $\to$ failure is simply the probability given by two successes and a failure. We can then find the probability that the next three events are two successes and one failure to get:

        $$\boxed{P[2,1]=(.95)^2(.05)=.045125}$$

      \item Since we know that an error is expected to occur after twenty attempts, we would expected for there to be another error twenty attempts after an initial one. As such, with a 2-second request time, we may simply write the time between errors as:

        $$\boxed{t=20(2)=40[\si{\second}]}$$

    \end{enumerate}

  \item 

    \begin{enumerate}

      \item We may express the CDF as:

        $$P[K\leq n]=\sum_{K}^n \left( \begin{matrix} n\\K\end{matrix} \right)p^k(1-p)^{n-k}$$

        We first find the expectation value:

        $$E[K]=np=6\left( \frac{1}{3} \right)$$
        $$\boxed{E[K]=2}$$

        From here, we plug in our information to write:

        $$P[K<2]=P[K=0]+P[K=1]$$

        This gives us:

        $$P[K< 2]=\left( \begin{matrix}6\\0\end{matrix} \right)\left( \frac{1}{3} \right)^0\left( \frac{2}{3} \right)^{6}+\left( \begin{matrix}6\\1\end{matrix} \right)\left( \frac{1}{3} \right)^1\left( \frac{2}{3} \right)^{5}$$
        $$P[K< 2]=.3512$$

        We then take the complement to get:

        $$\boxed{P[K\geq E[K]]=.6488}$$

      \item We are given Poisson with $\alpha=3$. This gives us:

        $$P_K(k)=\frac{3^k}{k!}e^{-3}$$

        Since we know $\alpha$ is the expectation value, we may write the probability we want to find as:

        $$P[K\geq 3]=1-P[K<3]$$

        Thus, we find $P[K<3]$:

        $$P[K<3]=\left[\frac{3^2}{2!}+\frac{3^1}{1!}+\frac{3^0}{0!}\right]e^{-3}$$
        $$P[K<3]=.4232$$

        We then take the complement to get:

        $$\boxed{P[K\geq 3]=.5768}$$

      \item Next, we are given a discrete uniform distribution, with $k=0$ and $l=4$. We begin by calculating the expectation value:

        $$E[K]=\frac{4+0}{2}=2$$

        Thus, we want to find:

        $$P[K\geq 2]=1-P[K=0]-P[K=1]$$

        We write the probability function as:

        $$P[K]=\frac{1}{n}=\frac{1}{l-k-1}=\frac{1}{3}$$

        We then get:

        $$P[K\geq 2]=1-2\left( \frac{1}{3} \right)$$
        $$\boxed{P[K\geq 2]=\frac{1}{3}}$$

    \end{enumerate}

  \item 

    \begin{enumerate}

      \item We may begin by defining the PMF using the transition points, or:

        $$x=-3\to (.25-0)=.25$$
        $$x=5\to (.6-.25)=.35$$
        $$x=7\to (1-.6)=.4$$

        Thus, we get:

        $$\boxed{P_X(x)=\left\{\begin{array}{ll} .25, & x=-3\\ .35, & x=5\\ .4, & x=7\\ 0, & \text{otherwise} \end{array}}$$

        \item Since $U=X^2/2$, we may convert the values to:

        $$U=4.5\to .25$$
        $$U=12.5\to .35$$
        $$U=24.5\to .4$$

        As such, we may write:

        $$\boxed{P_U(u)=\left\{\begin{array}{ll} .25, & U=4.5\\ .35, & U=12.5\\ .4, & U=24.5\\ 0, & \text{otherwise} \end{array}}$$

        The expectation value may be found as:

        $$E(U)=.25(4.5)+.35(12.5)+.4(24.5)$$
        $$\boxed{E(U)=15.3}$$

      \item Given that $V=2|X|$, we may get:

        $$V=6\to .25$$
        $$V=10\to .35$$
        $$V=14\to .4$$

        Which gives us:

        $$\boxed{P_V(v)=\left\{\begin{array}{ll} .25, & V=6\\ .35, & V=10\\ .4, & V=14\\ 0, & \text{otherwise} \end{array}}$$

        This gives us an expectation value of:

        $$E(V)=.25(6)+.35(10)+.4(14)$$
        $$\boxed{E(V)=10.6}$$

        Finally, we can calculate the variance as:

        $$\text{Var}(V)=.25(6-10.6)^2+.35(10-10.6)^2+.4(14-10.6)^2$$
        $$\boxed{\text{Var}(V)=10.6}$$

    \end{enumerate}

  \item 

    \begin{enumerate}

      \item We may compute the expectation value as:

        $$E[X]=\sum_{1}^4 XP_X(x)$$

        This gives us:

        $$E[X]=.1+2(.4)+3(.3)+4(.2)$$
        $$\boxed{E[X]=2.6}$$

      \item We can calculate the standard deviation as:

        $$\sigma_x=\sqrt{\sum_1^4 (X-E[X])^2P_X(x)}$$

        This gives us:

        $$\sigma_x=\sqrt{.1(1.6)^2+.4(.6)^2+.3(.4)^2+.2(1.4)^2}$$
        $$\boxed{\sigma_x=.9165}$$

      \item We know that $Y=200-4X$. Per our properties, we know that the expectation value is simply plugged into the equation to get:

        $$E[Y]=200-4E[X]$$
        $$\boxed{E[Y]=189.6}$$

        Standard deviation would only be multiplied by the factor to get:

        $$\sigma_Y=4\sigma_X$$
        $$\boxed{\sigma_Y=3.6661}$$

      \item We may simply plug the values in to get:

        $$W=0\to .1$$
        $$W=1.25\to .4$$
        $$W=5\to .3$$
        $$W=11.25\to .2$$

        We then calculate:

        $$E[W]=1.25(.4)+5(.3)+11.25(.2)$$
        $$\boxed{E[W]=4.25}$$

      \item We can calculate using the mean from (a) to get:

        $$1.25(2.6-1)^2=3.2$$

        The two do not equal each other. This is because $W$ changes $X$ by a non-linear operation (\textit{i}.\textit{e}.\ squaring)

    \end{enumerate}

  \item First and foremost, let us refer to the cost per device as $C$. This lets us construct an ``operational'' probability as:

    $$P_{\text{working, regular}}=(1-.08)^{10}=.4344$$
    $$P_{\text{working, reliable}}=(1-.02)^{10}=.8171$$

    Each module costs $C$ to manufacture. We can calculate each individual one as:

    $$C_{\text{regular}}=2(10)+2=22\$$$
    $$C_{\text{reliable}}=6(10)+2=62\$$$

    We can then express the revenue as the probability of success times the selling price ($X$), minus the manufacture price of the ten units:

    $$E_{\text{regular}}[R]=.4344X-22$$
    $$E_{\text{reliable}}[R]=.8171X-62$$


    We can calculate the point of equivalence as:

    $$.4344X-22=.8171X-62$$
    $$.3827X=40$$
    $$\boxed{X=104.52\approx 105}$$

    We can then check for which value there is more revenue before the equivalence point. For example, let us take $X=104$. This gives us:

    $$E_{\text{regular}}[104]=23.178$$
    $$E_{\text{reliable}}[104]=22.978$$

    Thus, we may conclude that, if $104$ or less units are constructed, more revenue will be generated if the less-reliable units are used. Conversely, if more than $104$ units are to be constructed, more revenue will be generated if using the reliable components.

    \setcounter{enumi}{9}

  \item 

    \begin{enumerate}

      \item This distribution can, as a matter of fact, be regarded as binomial, since the choices are binary: either the sensor is or is not the most efficient. If each sensor is equally likely to be chosen, and there are four sensors, then the probability of making the correct choice is:

        $$\boxed{p=\frac{1}{n}=\frac{1}{4}=.25}$$
        
        Conversely, the probability that the incorrect choice is made is:

        $$q=1-p=.75$$

        Since a sensor is chosen ten times, we may find that $\boxed{n=10}$

        Using the binomial distribution, we may calculate the probability that $3$ correct choices are made as:

        $$P_X[k=3]=\left( \begin{matrix}10\\3\end{matrix} \right)(.25)^3(.75)^7$$
        $$\boxed{P_X[k=3]=.2503}$$

      \item Since the success rate for each trial is not the same, this is no longer a binomial distribution. The probability of correct selection for three out of 5 of the remaining components becomes:

        $$P_{X'}[k=3]=\left( \begin{matrix}5\\3\end{matrix} \right)(.25)^3(.75)^2$$
        $$\boxed{P_{X'}[k=3]=.0879}$$

    \end{enumerate}

\end{enumerate}

\end{document}

