%%%%%%%%%%%%%%%%%%%%%%%%%%%%%%%%%%%%%%%%%%%%%%%%%%%%%%%%%%%%%%%%%%%%%%%%%%%%%%%%%%%%%%%%%%%%%%%%%%%%%%%%%%%%%%%%%%%%%%%%%%%%%%%%%%%%%%%%%%%%%%%%%%%%%%%%%%%%%%%%%%%
% Written By Michael Brodskiy
% Class: Analysis of Random Phenomena
% Professor: I. Salama
%%%%%%%%%%%%%%%%%%%%%%%%%%%%%%%%%%%%%%%%%%%%%%%%%%%%%%%%%%%%%%%%%%%%%%%%%%%%%%%%%%%%%%%%%%%%%%%%%%%%%%%%%%%%%%%%%%%%%%%%%%%%%%%%%%%%%%%%%%%%%%%%%%%%%%%%%%%%%%%%%%%

\documentclass[12pt]{article} 
\usepackage{alphalph}
\usepackage[utf8]{inputenc}
\usepackage[russian,english]{babel}
\usepackage{titling}
\usepackage{float}
\usepackage{amsmath}
\usepackage{graphicx}
\usepackage{enumitem}
\usepackage{amssymb}
\usepackage[super]{nth}
\usepackage{everysel}
\usepackage{ragged2e}
\usepackage{geometry}
\usepackage{multicol}
\usepackage{fancyhdr}
\usepackage{cancel}
\usepackage{siunitx}
\usepackage{physics}
\usepackage{tikz}
\usepackage{mathdots}
\usepackage{yhmath}
\usepackage{cancel}
\usepackage{color}
\usepackage{xcolor}
\usepackage{colortbl}
\usepackage{array}
\usepackage{multirow}
\usepackage{gensymb}
\usepackage{tabularx}
\usepackage{extarrows}
\usepackage{booktabs}
\usepackage{lastpage}
\usetikzlibrary{fadings}
\usetikzlibrary{patterns}
\usetikzlibrary{shadows.blur}
\usetikzlibrary{shapes}

\geometry{top=1.0in,bottom=1.0in,left=1.0in,right=1.0in}
\newcommand{\subtitle}[1]{%
  \posttitle{%
    \par\end{center}
    \begin{center}\large#1\end{center}
    \vskip0.5em}%

}
\usepackage{hyperref}
\hypersetup{
colorlinks=true,
linkcolor=blue,
filecolor=magenta,      
urlcolor=blue,
citecolor=blue,
}


\title{Homework 6}
\date{\today}
\author{Michael Brodskiy\\ \small Professor: I. Salama}

\begin{document}

\maketitle

\begin{enumerate}

  \item We begin by sketching the CDF:

    \begin{figure}[H]
      \centering
      \tikzset{every picture/.style={line width=0.75pt}} %set default line width to 0.75pt        

\begin{tikzpicture}[x=0.75pt,y=0.75pt,yscale=-1,xscale=1]
%uncomment if require: \path (0,346); %set diagram left start at 0, and has height of 346

%Shape: Axis 2D [id:dp37978310655082725] 
\draw  (126,230.7) -- (475,230.7)(160.9,66) -- (160.9,249) (468,225.7) -- (475,230.7) -- (468,235.7) (155.9,73) -- (160.9,66) -- (165.9,73)  ;
%Straight Lines [id:da3909651706663365] 
\draw    (220.9,225.28) -- (220.9,235.7) ;
%Straight Lines [id:da6155381643039015] 
\draw    (280.9,225.28) -- (280.9,235.7) ;
%Straight Lines [id:da48285074488447755] 
\draw    (340.9,225.28) -- (340.9,235.7) ;
%Straight Lines [id:da6610074909520481] 
\draw    (400.9,225.28) -- (400.9,235.7) ;
%Straight Lines [id:da8426487929903638] 
\draw    (460.9,225.28) -- (460.9,235.7) ;
%Straight Lines [id:da1269344154444887] 
\draw    (462.37,102.15) -- (220.9,195.28) ;
%Straight Lines [id:da6098071503367037] 
\draw    (220.9,235.7) -- (220.9,195.28) ;
%Straight Lines [id:da7389828699591082] 
\draw  [dash pattern={on 4.5pt off 4.5pt}]  (220.9,195.28) -- (161.48,195.28) ;
%Straight Lines [id:da27575851026646836] 
\draw  [dash pattern={on 4.5pt off 4.5pt}]  (402.16,125.01) -- (160.48,125.28) ;
%Straight Lines [id:da5705188430777732] 
\draw    (462.37,102.15) -- (462.16,89.28) ;
%Straight Lines [id:da12116100558239962] 
\draw  [dash pattern={on 4.5pt off 4.5pt}]  (462.16,89.28) -- (160.74,89.28) ;
%Straight Lines [id:da42590383894759165] 
\draw    (475.59,89.28) -- (462.16,89.28) ;
\draw [shift={(478.59,89.28)}, rotate = 180] [fill={rgb, 255:red, 0; green, 0; blue, 0 }  ][line width=0.08]  [draw opacity=0] (8.93,-4.29) -- (0,0) -- (8.93,4.29) -- cycle    ;
%Straight Lines [id:da3414535728667185] 
\draw    (400.9,225.28) -- (401.16,125.01) ;

% Text Node
\draw (162.23,56) node [anchor=south] [inner sep=0.75pt]    {$F_{X}( x)$};
% Text Node
\draw (477,233.4) node [anchor=north west][inner sep=0.75pt]    {$X$};
% Text Node
\draw (220.9,239.1) node [anchor=north] [inner sep=0.75pt]    {$1$};
% Text Node
\draw (280.9,239.1) node [anchor=north] [inner sep=0.75pt]    {$2$};
% Text Node
\draw (340.9,239.1) node [anchor=north] [inner sep=0.75pt]    {$3$};
% Text Node
\draw (400.9,239.1) node [anchor=north] [inner sep=0.75pt]    {$4$};
% Text Node
\draw (460.9,239.1) node [anchor=north] [inner sep=0.75pt]    {$5$};
% Text Node
\draw (158.9,234.1) node [anchor=north east] [inner sep=0.75pt]    {$0$};
% Text Node
\draw (158.9,89.28) node [anchor=east] [inner sep=0.75pt]    {$1$};
% Text Node
\draw (158.9,195.34) node [anchor=east] [inner sep=0.75pt]    {$.25$};
% Text Node
\draw (158.9,159.99) node [anchor=east] [inner sep=0.75pt]    {$.5$};
% Text Node
\draw (158.9,124.63) node [anchor=east] [inner sep=0.75pt]    {$.75$};


\end{tikzpicture}

      \caption{CDF of Given Function}
      \label{fig:1}
    \end{figure}

    \begin{enumerate}

      \item From the CDF, we may conclude:

        $$P[X<1]=F_X(X^-=1)-F_X(X=0)$$
        $$\boxed{P[X<1]=0}$$

        And then:

        $$P[X\leq 1]=F_X[X=1]-F_X[X=0]$$
        $$P[X\leq 1]=.25-0$$
        $$\boxed{P[X\leq 1]=.25}$$

      \item Since the given probability occurs in the continuous portion, we can find that:

        $$P[X<2]=P[X\leq 2]$$

        Furthermore, we can write:

        $$P[X\leq 2]=F_X[X=2]-F_X[X=0]$$
        $$P[X\leq 2]=\frac{25}{60}-0$$
        $$\boxed{P[X\leq 2]=P[X<2]=.416\bar{6}}$$

      \item We continue to find:

        $$P[X>5]=1-F_X(X^-=5)$$
        $$P[X>5]=1-.916\bar{6}$$
        $$\boxed{P[X>5]=.083\bar{3}}$$

        Since the right side is not in the continuous portion, we get:

        $$P[X\geq5]=1-F_X(X^+=5)$$
        $$P[X\geq5]=1-1$$
        $$\boxed{P[X\geq5]=0}$$

      \item With the given range, we may find:

        $$P[1<X<2]=F_X(X^-=2)-F_X(X^+=1)$$
        $$P[1<X<2]=.416\bar{6}-.25$$
        $$\boxed{P[1<X<2]=.16\bar{6}}$$

    \end{enumerate}

  \item

    \begin{enumerate}

      \item We may begin by finding a PDF for the continuous region:

        $$f_{W,2}(w)=(1-f_{W,1})\lambda e^{-\lambda w}$$

        Furthermore, we know that:

        $$f_{W,1}(w)=.3$$

        Given the expectation value given successful communications, we may know:

        $$E[W]=\frac{1}{\lambda}$$

        This gives us:

        $$\lambda=\frac{1}{3}[\si{min}]$$

        Thus, the pdf becomes:

        $$\boxed{f_{W}(w)=\left\{\begin{array}{ll} \dfrac{3}{10}, & w=0\\\\ \dfrac{7}{30}e^{-\frac{1}{3}w} & w>0 \end{array}}$$

      \item We may find the CDF by integrating to get:

        $$F_{W}(w)=\left\{\begin{array}{ll} \dfrac{3}{10}, & w=0\\\\ 1-.7e^{-\frac{1}{3}w} & w>0 \end{array}$$

        We may note that this is continuous, so we may simply write:

        $$\boxed{F_{W}(w)=1-.7e^{-\frac{1}{3}w}\quad w\geq 0}$$

      \item We can write the expectation value as:

        $$E[W]=\int_{0}^{\infty} wf_W(w)\,dw$$

        This gives us:

        $$E[W]=.3(w\to0)+\int_{0^+}^{\infty} \frac{7w}{30}e^{-\frac{1}{3}w}\,dw$$

        We evaluate to get:

        $$E[W]=.3(0)+2.1$$
        $$\boxed{E[W]=2.1[\si{min}]}$$

        We can then find the variance using the law of total variance as:

        $$\text{Var}[W]=E[\text{Var}[W|S]]+\text{Var}[E[W|S]]$$

        Where $S$ corresponds to success/failure. Thus, we use the information given:

        $$E[W|S]=3\quad\text{ and }\quad \text{Var}[W|S]=\frac{1}{\lambda^2}=9$$

        From here, we can get:

        $$\text{Var}[E[W|X]]=(3-2.1)^2\cdot S\cdot S^c$$
        $$\text{Var}[E[W|X]]=(3-2.1)^2\cdot (.7)\cdot (.3)$$
        $$\text{Var}[E[W|X]]=.1701$$

        This gives us:

        $$\text{Var}[W]=(9)(.7)+.1701$$
        $$\boxed{\text{Var}[W]=6.4701[\si{min\squared}]}$$

      \item To define this probability, we simply use the CDF to write:

        $$P[W<1]=F_W(w=1^-)$$

        We can find this using:

        $$P[W<1]=1-.7e^{-1/3}$$
        $$\boxed{P[W<1]=.4984}$$

    \end{enumerate}

  \item

    \begin{enumerate}

      \item 

      \item 

      \item 

      \item 

    \end{enumerate}

  \item

    \begin{enumerate}

      \item 

      \item 

      \item 

      \item 

      \item 

      \item 

      \item 

    \end{enumerate}

  \item

    \begin{enumerate}

      \item 

      \item 

      \item 

      \item 

      \item 

    \end{enumerate}

  \item

    \begin{enumerate}

      \item 

      \item 

      \item 

      \item 

    \end{enumerate}

  \item

    \begin{enumerate}

      \item 

      \item 

      \item 

    \end{enumerate}

\end{enumerate}

\end{document}

