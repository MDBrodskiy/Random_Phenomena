%%%%%%%%%%%%%%%%%%%%%%%%%%%%%%%%%%%%%%%%%%%%%%%%%%%%%%%%%%%%%%%%%%%%%%%%%%%%%%%%%%%%%%%%%%%%%%%%%%%%%%%%%%%%%%%%%%%%%%%%%%%%%%%%%%%%%%%%%%%%%%%%%%%%%%%%%%%%%%%%%%%
% Written By Michael Brodskiy
% Class: Analysis of Random Phenomena
% Professor: I. Salama
%%%%%%%%%%%%%%%%%%%%%%%%%%%%%%%%%%%%%%%%%%%%%%%%%%%%%%%%%%%%%%%%%%%%%%%%%%%%%%%%%%%%%%%%%%%%%%%%%%%%%%%%%%%%%%%%%%%%%%%%%%%%%%%%%%%%%%%%%%%%%%%%%%%%%%%%%%%%%%%%%%%

\documentclass[12pt]{article} 
\usepackage{alphalph}
\usepackage[utf8]{inputenc}
\usepackage[russian,english]{babel}
\usepackage{titling}
\usepackage{float}
\usepackage{amsmath}
\usepackage{graphicx}
\usepackage{enumitem}
\usepackage{amssymb}
\usepackage[super]{nth}
\usepackage{everysel}
\usepackage{ragged2e}
\usepackage{geometry}
\usepackage{multicol}
\usepackage{fancyhdr}
\usepackage{cancel}
\usepackage{siunitx}
\usepackage{physics}
\usepackage{tikz}
\usepackage{mathdots}
\usepackage{yhmath}
\usepackage{cancel}
\usepackage{color}
\usepackage{xcolor}
\usepackage{colortbl}
\usepackage{array}
\usepackage{multirow}
\usepackage{gensymb}
\usepackage{tabularx}
\usepackage{extarrows}
\usepackage{booktabs}
\usepackage{lastpage}
\usetikzlibrary{fadings}
\usetikzlibrary{patterns}
\usetikzlibrary{shadows.blur}
\usetikzlibrary{shapes}

\geometry{top=1.0in,bottom=1.0in,left=1.0in,right=1.0in}
\newcommand{\subtitle}[1]{%
  \posttitle{%
    \par\end{center}
    \begin{center}\large#1\end{center}
    \vskip0.5em}%

}
\usepackage{hyperref}
\hypersetup{
colorlinks=true,
linkcolor=blue,
filecolor=magenta,      
urlcolor=blue,
citecolor=blue,
}


\title{Homework 6}
\date{\today}
\author{Michael Brodskiy\\ \small Professor: I. Salama}

\begin{document}

\maketitle

\begin{enumerate}

  \item We begin by sketching the CDF:

    \begin{figure}[H]
      \centering
      \tikzset{every picture/.style={line width=0.75pt}} %set default line width to 0.75pt        

\begin{tikzpicture}[x=0.75pt,y=0.75pt,yscale=-1,xscale=1]
%uncomment if require: \path (0,346); %set diagram left start at 0, and has height of 346

%Shape: Axis 2D [id:dp37978310655082725] 
\draw  (126,230.7) -- (475,230.7)(160.9,66) -- (160.9,249) (468,225.7) -- (475,230.7) -- (468,235.7) (155.9,73) -- (160.9,66) -- (165.9,73)  ;
%Straight Lines [id:da3909651706663365] 
\draw    (220.9,225.28) -- (220.9,235.7) ;
%Straight Lines [id:da6155381643039015] 
\draw    (280.9,225.28) -- (280.9,235.7) ;
%Straight Lines [id:da48285074488447755] 
\draw    (340.9,225.28) -- (340.9,235.7) ;
%Straight Lines [id:da6610074909520481] 
\draw    (400.9,225.28) -- (400.9,235.7) ;
%Straight Lines [id:da8426487929903638] 
\draw    (460.9,225.28) -- (460.9,235.7) ;
%Straight Lines [id:da1269344154444887] 
\draw    (462.37,102.15) -- (220.9,195.28) ;
%Straight Lines [id:da6098071503367037] 
\draw    (220.9,235.7) -- (220.9,195.28) ;
%Straight Lines [id:da7389828699591082] 
\draw  [dash pattern={on 4.5pt off 4.5pt}]  (220.9,195.28) -- (161.48,195.28) ;
%Straight Lines [id:da27575851026646836] 
\draw  [dash pattern={on 4.5pt off 4.5pt}]  (402.16,125.01) -- (160.48,125.28) ;
%Straight Lines [id:da5705188430777732] 
\draw    (462.37,102.15) -- (462.16,89.28) ;
%Straight Lines [id:da12116100558239962] 
\draw  [dash pattern={on 4.5pt off 4.5pt}]  (462.16,89.28) -- (160.74,89.28) ;
%Straight Lines [id:da42590383894759165] 
\draw    (475.59,89.28) -- (462.16,89.28) ;
\draw [shift={(478.59,89.28)}, rotate = 180] [fill={rgb, 255:red, 0; green, 0; blue, 0 }  ][line width=0.08]  [draw opacity=0] (8.93,-4.29) -- (0,0) -- (8.93,4.29) -- cycle    ;
%Straight Lines [id:da3414535728667185] 
\draw    (400.9,225.28) -- (401.16,125.01) ;

% Text Node
\draw (162.23,56) node [anchor=south] [inner sep=0.75pt]    {$F_{X}( x)$};
% Text Node
\draw (477,233.4) node [anchor=north west][inner sep=0.75pt]    {$X$};
% Text Node
\draw (220.9,239.1) node [anchor=north] [inner sep=0.75pt]    {$1$};
% Text Node
\draw (280.9,239.1) node [anchor=north] [inner sep=0.75pt]    {$2$};
% Text Node
\draw (340.9,239.1) node [anchor=north] [inner sep=0.75pt]    {$3$};
% Text Node
\draw (400.9,239.1) node [anchor=north] [inner sep=0.75pt]    {$4$};
% Text Node
\draw (460.9,239.1) node [anchor=north] [inner sep=0.75pt]    {$5$};
% Text Node
\draw (158.9,234.1) node [anchor=north east] [inner sep=0.75pt]    {$0$};
% Text Node
\draw (158.9,89.28) node [anchor=east] [inner sep=0.75pt]    {$1$};
% Text Node
\draw (158.9,195.34) node [anchor=east] [inner sep=0.75pt]    {$.25$};
% Text Node
\draw (158.9,159.99) node [anchor=east] [inner sep=0.75pt]    {$.5$};
% Text Node
\draw (158.9,124.63) node [anchor=east] [inner sep=0.75pt]    {$.75$};


\end{tikzpicture}

      \caption{CDF of Given Function}
      \label{fig:1}
    \end{figure}

    \begin{enumerate}

      \item From the CDF, we may conclude:

        $$P[X<1]=F_X(X^-=1)-F_X(X=0)$$
        $$\boxed{P[X<1]=0}$$

        And then:

        $$P[X\leq 1]=F_X[X=1]-F_X[X=0]$$
        $$P[X\leq 1]=.25-0$$
        $$\boxed{P[X\leq 1]=.25}$$

      \item Since the given probability occurs in the continuous portion, we can find that:

        $$P[X<2]=P[X\leq 2]$$

        Furthermore, we can write:

        $$P[X\leq 2]=F_X[X=2]-F_X[X=0]$$
        $$P[X\leq 2]=\frac{25}{60}-0$$
        $$\boxed{P[X\leq 2]=P[X<2]=.416\bar{6}}$$

      \item We continue to find:

        $$P[X>5]=1-F_X(X^-=5)$$
        $$P[X>5]=1-.916\bar{6}$$
        $$\boxed{P[X>5]=.083\bar{3}}$$

        Since the right side is not in the continuous portion, we get:

        $$P[X\geq5]=1-F_X(X^+=5)$$
        $$P[X\geq5]=1-1$$
        $$\boxed{P[X\geq5]=0}$$

      \item With the given range, we may find:

        $$P[1<X<2]=F_X(X^-=2)-F_X(X^+=1)$$
        $$P[1<X<2]=.416\bar{6}-.25$$
        $$\boxed{P[1<X<2]=.16\bar{6}}$$

    \end{enumerate}

  \item

    \begin{enumerate}

      \item We may begin by finding a PDF for the continuous region:

        $$f_{W,2}(w)=(1-f_{W,1})\lambda e^{-\lambda w}$$

        Furthermore, we know that:

        $$f_{W,1}(w)=.3$$

        Given the expectation value given successful communications, we may know:

        $$E[W]=\frac{1}{\lambda}$$

        This gives us:

        $$\lambda=\frac{1}{3}[\si{min}]$$

        Thus, the pdf becomes:

        $$\boxed{f_{W}(w)=\left\{\begin{array}{ll} \dfrac{3}{10}, & w=0\\\\ \dfrac{7}{30}e^{-\frac{1}{3}w} & w>0 \end{array}}$$

      \item We may find the CDF by integrating to get:

        $$F_{W}(w)=\left\{\begin{array}{ll} \dfrac{3}{10}, & w=0\\\\ 1-.7e^{-\frac{1}{3}w} & w>0 \end{array}$$

        We may note that this is continuous, so we may simply write:

        $$\boxed{F_{W}(w)=1-.7e^{-\frac{1}{3}w}\quad w\geq 0}$$

      \item We can write the expectation value as:

        $$E[W]=\int_{0}^{\infty} wf_W(w)\,dw$$

        This gives us:

        $$E[W]=.3(w\to0)+\int_{0^+}^{\infty} \frac{7w}{30}e^{-\frac{1}{3}w}\,dw$$

        We evaluate to get:

        $$E[W]=.3(0)+2.1$$
        $$\boxed{E[W]=2.1[\si{min}]}$$

        We can then find the variance using:

        $$\text{Var}[W]=E[W^2]-E[W]^2$$

        From here, we can get:

        $$\text{Var}[W]=\int_0^{\infty} \frac{7w^2}{30}e^{-\frac{1}{3}w}\,dw-2.1^2$$

        This gives us:

        $$\text{Var}[W]=12.6-4.41$$
        $$\boxed{\text{Var}[W]=8.19[\si{min\squared}]}$$

      \item To define this probability, we simply use the CDF to write:

        $$P[W<1]=F_W(w=1^-)$$

        We can find this using:

        $$P[W<1]=1-.7e^{-1/3}$$
        $$\boxed{P[W<1]=.4984}$$

    \end{enumerate}

  \item

    \begin{enumerate}

      \item We know that the expression for the joint PMF relies on $X$, which can be $0$ to $3$ unsuccessful attempts, and $Y$, which can be $0$ to $2$ attempts to success after the second attempt. Thus, we may express the joint PMF as a matrix:

        $$P_{XY}(x,y)=\left\{ \begin{matrix} .512 & .384 & .096 & .008\\ 0 & .096 & .096 & .024\\ 0 & 0 & 0 & .008\end{matrix} \right\}$$

      \item We proceed to find the marginal PMFs as:

        $$P_X(x)=\sum_y P_{XY}(x,y)$$

        This gives us:

        $$P_{X}(x)=\left\{ \begin{matrix} .512 + 0 + 0 + 0 \\ .384 + .096\\ .096\cdot2\\ .024+2\cdot.008\end{matrix} \right\}$$
        $$\boxed{P_{X}(x)=\left\{ \begin{matrix} .512\\ .48\\ .192\\ .04\end{matrix} \right\}}$$

        Similarly, we may find the marginal PMF of $Y$ as:

        $$P_Y(y)=\sum_x P_{XY}(x,y)$$

        This gives us:

        $$P_{Y}(y)=\left\{ \begin{matrix} .512+.384+.096+.008 \\ .024+2\cdot.096\\ .008\end{matrix} \right\}$$
        $$\boxed{P_{Y}(y)=\left\{ \begin{matrix} 1\\ .216 \\ .008\end{matrix} \right\}}$$

      \item We can find this using the formula:

        $$P_{X|Y}(x|y=1)=\frac{P_{XY}(x,1)}{P_Y(1)}$$

        Using the results from (b), we obtain:

        $$P_{X|Y}(x|y=1)=\frac{\left\{ \begin{matrix} 0\\.096\\.096\\.024\end{matrix} \right\}}{.216}$$

        We then get:

        $$\boxed{P_{X|Y}(x|y=1)=\left\{ \begin{matrix} 0\\.444\bar{4}\\.444\bar{4}\\.111\bar{1}\end{matrix} \right\}}$$

      \item Using the result from (c), we get:

        $$E[X|Y=1]=0\cdot 0 + 1\cdot .44\bar{4} + 2\cdot .44\bar{4}+3\cdot .11\bar{1}$$

        This becomes:

        $$\boxed{E[X|Y=1]=1.66\bar{6}}$$

    \end{enumerate}

  \item

    \begin{enumerate}

      \item Per the given joint PMF, we may find:

        $$\boxed{P_{X_1X_2}(x_1=2,x_2=2)=.1}$$

      \item We may sum along the diagonal to get:

        $$P(A\Rightarrow X_1=X_2)=.07+.15+.1+.07$$
        $$\boxed{P(A\Rightarrow X_1=X_2)=.39}$$

      \item First, we may express this as:

        $$P[B]=P(|X_1-X_2|\geq 3)$$

        This gets us the combinations of:

        $$(X_1,X_2)\Rightarrow (0,3), (0,4), (1,4), (3,0)$$

        We then sum each probability to get:

        $$P(|X_1-X_2|\geq 3)=.01+.02+.01+0$$
        $$\boxed{P(|X_1-X_2|\geq 3)=.04}$$

      \item First, we express this as:

        $$P[\text{tasks} =4]=P(X_1+X_2=4)$$

        Which gives us:

        $$P(X_1+X_2=4)=.1+.03+.04+.01$$
        $$\boxed{P(X_1+X_2=4)=.18}$$

        We can then find:
        
        $$P[\text{tasks} \geq4]=P(X_1+X_2=4)+P(X_1+X_2>4)$$

        As such, we get:

        $$P(X_1+X_2\geq4)=.18 + .06+.02+.04+.04+.07+.05$$
        $$\boxed{P(X_1+X_2\geq4)=.46}$$

      \item We can find the marginal PMF of $X_1$ by summing across the rows of $X_2$ to get:

        $$P_{X_1}(x_1)=\left\{ \begin{matrix}.07+.06+.04+0+.1\\ .08+.15+.05+.03+.02\\ .03+.05+.1+.04+.04\\ .01+.04+.06+.07+.05 \end{matrix} \right\}$$

        We evaluate to find:

        $$\boxed{P_{X_1}(x_1)=\left\{ \begin{matrix}.27\\ .33\\ .26\\ .23 \end{matrix} \right\}}$$

      \item Using the results from part (e), we may write the conditional marginal PMF as:

        $$P_{X_2|X_1=2}(x_2)=\frac{P_{X_2}(X_1=2,x_2)}{P_{X_1}(2)}$$

        This gives us:

        $$P_{X_2|X_1=2}(x_2)=\frac{\left\{ \begin{matrix}.03\\.05\\.1\\.04\\.04 \end{matrix}\right\}}{.26}$$
        $$\boxed{P_{X_2|X_1=2}(x_2)=\left\{ \begin{matrix}.1154\\.1923\\.3846\\.1538\\.1538 \end{matrix}\right\}}$$

      \item First, we find each value:

        $$P(X_1=0)=.18$$
        $$P(X_2=3)=.22$$
        $$P(X_1=0,X_2=3)=0$$

        Since $\boxed{P(X_1=0,X_2=3)\neq P(X_1=0)P(X_2=3)}$, the two are dependent.

    \end{enumerate}

  \item

    \begin{enumerate}

      \item Since $k$ ranges from $0$ to $n$, we simply drop the $k$-dependent terms to get:

        $$\boxed{P_N(n)=}\frac{5^n}{n!}e^{-5}$$

        We may observe that the expected rate is $\boxed{\lambda=5}$ packets per minute.

      \item We may write the conditional PMF as:

        $$P_{K|N}(k|n)=\frac{P_{NK}(n,k)}{P_N(n)}$$

        We plug both functions in to get:

        $$\boxed{P_{K|N}(k|n)=\left( \begin{matrix} n\\k\end{matrix} \right)p^k(1-p)^{n-k}}$$

      \item We know that an event is independent if:

        $$P_{NK}(n,k)=P_N(n)P_K(k)$$

        Since the $n$ choose $k$ function can not be separated into a function of $n$ and a function of $k$, we see that \underline{the events are dependent}

      \item We can calculate this as:

        $$P[N<3,K=1]=P_{NK}(0,1)+P_{NK}(1,1)+P_{NK}(2,1)$$

        We take $p=.8$ to write:

        $$P[N<3,K=1]=\sum_{n=0}^2 \frac{5^n}{n!}e^{-5}\left( \begin{matrix}n\\1\end{matrix} \right)(.8)(.2)^{n-1}$$

        This gives us:

        $$\boxed{P[N<3,K=1]=.053904}$$

      \item Using the conditional PMF obtained in (b), we get:

        $$E[K|N=10]=\sum_{k=0}^{10}\left( \begin{matrix}10\\k\end{matrix} \right)(.8)^k(.2)^{10-k}$$
        $$\boxed{E[K|N=10]=8[\si{packets}]}$$

    \end{enumerate}

  \item

    \begin{enumerate}

      \item We can find the range by finding the minimum and maximum amount of files to be uploaded. First, we know that it is possible that there are no users waiting to upload a file. In such a case, we have $Y=0$. On the other hand, it is possible that there are 3 users waiting to upload 3 files each, which gives us: $Y=3^2=9$. Thus, we find the range to be:

        $$\boxed{Y=[0,9]}$$

      \item To find this, we may write:

        $$P(X=3,Y=3)=P(X=3)\cdot P(Y=3|X=3)$$

        The following is given:

        $$P(X=3)=.3$$

        We then find:

        $$P(Y=3|X=3)= (.5)^3$$
        $$P(Y=3|X=3)= .125$$

        This gives us:

        $$P(X=3,Y=3)=(.3)(.125)$$
        $$\boxed{P(X=3,Y=3)=.0375}$$

      \item Similarly, we find:

        $$P(Y=7|X=3)= 3[(.1)^2(.5)+(.1)(.4)^2]$$
        $$P(Y=7|X=3)= .063$$

        From here, we get:

        $$P(X=3,Y=7)=(.3)(.063)$$
        $$\boxed{P(X=3,Y=7)=.0189}$$

      \item Since there are now exactly 2 users, this means there are, at most, 3 files each in queue, and at least 1, which gives us:

        $$\boxed{Y=[2,6]}$$

        Given the two users, we can iterate through to calculate the PMF as:

        $$P_Y(y)=\left\{ \begin{matrix} (.5)^2\\ (2)(.5)(.4)\\ (.4)^2+2(.1)(.5)\\ (2)(.4)(.1)\\ (.1)^2\end{matrix} \right\}$$

        Evaluating, this gives us:

        $$\boxed{P_Y(y)=\left\{ \begin{matrix} .25\\.4\\ .26\\ .08\\ .01\end{matrix} \right\}}$$

        We then compute the expectation value as:

        $$E[Y|X=2]=2(.25)+3(.4)+4(.26)+5(.08)+6(.01)$$
        $$\boxed{E[Y|X=2]=3.2}$$

    \end{enumerate}

  \item

    \begin{enumerate}

      \item We can integrate over the interval to write:

        $$\int_{-1}^1\int_{0}^{x^2} cx^2\,dy\,dx=1$$

        We evaluate to get:

        $$\int_{-1}^1 cx^4\,dx=1$$
        $$\frac{cx^5}{5}\Big_{-1}^1=1$$
        $$\frac{2c}{5}=1$$
        $$\boxed{c=\frac{5}{2}}$$

        Thus, we may write:

        $$f_{X,Y}(x,y)=\left\{ \begin{array}{lll} 2.5x^2, & -1\leq x\leq 1, & 0\leq y\leq x^2\\ 0, & \text{otherwise}\end{array}$$

      \item We can find the marginal PDF with respect to $X$ as:

        $$f_X(x)=\int_0^{x^2} cx^2\,dy$$

        This gives us:

        $$\boxed{f_X(x)=\frac{5}{2}x^4,\quad -1\leq x\leq 1}$$

      \item Similarly, we may find the marginal PDF with respect to $Y$ as:

        $$f_Y(y)=\int_{-\sqrt{y}}^{\sqrt{y}} cx^2\,dx$$
        $$f_Y(y)=\frac{c}{3}\left[ x^3 \right]\Big_{-\sqrt{y}}^{\sqrt{y}}$$
        $$\boxed{f_Y(y)=\frac{5}{3}y^{3/2},\quad 0\leq y\leq 1}$$

    \end{enumerate}

\end{enumerate}

\end{document}

