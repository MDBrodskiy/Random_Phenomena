%%%%%%%%%%%%%%%%%%%%%%%%%%%%%%%%%%%%%%%%%%%%%%%%%%%%%%%%%%%%%%%%%%%%%%%%%%%%%%%%%%%%%%%%%%%%%%%%%%%%%%%%%%%%%%%%%%%%%%%%%%%%%%%%%%%%%%%%%%%%%%%%%%%%%%%%%%%%%%%%%%%
% Written By Michael Brodskiy
% Class: Analysis of Random Phenomena
% Professor: I. Salama
%%%%%%%%%%%%%%%%%%%%%%%%%%%%%%%%%%%%%%%%%%%%%%%%%%%%%%%%%%%%%%%%%%%%%%%%%%%%%%%%%%%%%%%%%%%%%%%%%%%%%%%%%%%%%%%%%%%%%%%%%%%%%%%%%%%%%%%%%%%%%%%%%%%%%%%%%%%%%%%%%%%

\documentclass[12pt]{article} 
\usepackage{alphalph}
\usepackage[utf8]{inputenc}
\usepackage[russian,english]{babel}
\usepackage{titling}
\usepackage{float}
\usepackage{amsmath}
\usepackage{graphicx}
\usepackage{enumitem}
\usepackage{amssymb}
\usepackage[super]{nth}
\usepackage{everysel}
\usepackage{ragged2e}
\usepackage{geometry}
\usepackage{multicol}
\usepackage{fancyhdr}
\usepackage{cancel}
\usepackage{siunitx}
\usepackage{physics}
\usepackage{tikz}
\usepackage{mathdots}
\usepackage{yhmath}
\usepackage{cancel}
\usepackage{color}
\usepackage{xcolor}
\usepackage{colortbl}
\usepackage{array}
\usepackage{multirow}
\usepackage{gensymb}
\usepackage{tabularx}
\usepackage{extarrows}
\usepackage{booktabs}
\usepackage{lastpage}
\usetikzlibrary{fadings}
\usetikzlibrary{patterns}
\usetikzlibrary{shadows.blur}
\usetikzlibrary{shapes}

\geometry{top=1.0in,bottom=1.0in,left=1.0in,right=1.0in}
\newcommand{\subtitle}[1]{%
  \posttitle{%
    \par\end{center}
    \begin{center}\large#1\end{center}
    \vskip0.5em}%

}
\usepackage{hyperref}
\hypersetup{
colorlinks=true,
linkcolor=blue,
filecolor=magenta,      
urlcolor=blue,
citecolor=blue,
}


\title{Homework 9}
\date{\today}
\author{Michael Brodskiy\\ \small Professor: I. Salama}

\begin{document}

\maketitle

\begin{enumerate}

  \item

    \begin{enumerate}

      \item Since it is given that $X,Y\in[0,1]$, we find that the range of $W=\text{Max}(X,Y)\to \boxed{W\in[0,1]}$. Therefore, we write:

        $$\boxed{S_W=[0,1]}$$

      \item Given that $w$ is the upper bound of $x$ and/or $y$, we may write:

        $$F_W(w)=\int_0^w\int_0^w 4xy\,dx\,dy$$

        We evaluate this to get:

        $$F_W(w)=4\int_0^wy\int_0^w x\,dx\,dy$$
        $$F_W(w)=4\left[\frac{y^2}{2}\right]\Big|_0^w\left[ \frac{x^2}{2} \right]\Big|_0^w$$
        $$\boxed{F_W(w)=\left\{ \begin{array}{ll} w^4, & 0\leq w<1\\ 1, & w\geq 1\\ 0, & \text{otherwise}\end{array}}$$

        From here, we know that:

        $$f_W(w)=\frac{d}{dw}[F_W(w)]$$

        This gives us:

        $$\boxed{f_W(w)=4w^3,\quad 0\leq w<1}$$

    \end{enumerate}

  \item

    \begin{enumerate}

      \item Given that $y\geq x$, we know that the lowest point for $W$ will occur when $y=x$, or $W=0$. Similarly, the biggest difference occurs when $y=1$ and $x=0$, which produces $W=1$. Thus, we write:

        $$\boxed{S_W=[0,1]}$$

      \item From here, we may write:

        $$F_W(w)=\int_0^1\int_0^{y-w} 6x\,dx\,dy$$

        We evaluate to get:

        $$F_W(w)=6\int_0^1[x^2/2]\Big|_0^{y-w}\,dy$$
        $$F_W(w)=3\int_0^1(y-w)^2\,dy$$
        $$F_W(w)=3\int_0^1y^2-2yw+w^2\,dy$$
        $$F_W(w)=y^3-3y^2w+3w^2y\Big|_0^1$$
        $$\boxed{F_W(w)=\left\{\begin{array}{ll} 1-3w+3w^2, & 0\leq w\leq1\\ 1, & w>1\\ 0, & \text{otherwise}\end{array}}$$

        We then differentiate to get:

        $$\boxed{f_W(w)=6w-3,\quad 0\leq w\leq 1}$$

    \end{enumerate}

  \item

    \begin{enumerate}

      \item We can express the PDF of $Y$ using the law of total probability as:

        $$f_Y(y)=P[Z=1]f_x(y)+P[Z=-1]f_x(-y)$$

        Substituting our values, since we know that $X$ is normal with $\mu=0$ and $\sigma^2=1$, we get:

        $$f_Y(y)=\frac{1}{\sqrt{2\pi}}\left[ pe^{-\frac{y^2}{2}}+(1-p)e^{-\frac{y^2}{2}} \right]$$

        Distributing and simplifying gives us:

        $$\boxed{f_Y(y)=\frac{1}{\sqrt{2\pi}}e^{-\frac{y^2}{2}}}$$

        Thus, we see that this returns to simply being a normal distribution.

      \item When we take $\mu\to 10$, this gives us:

        $$f_Y(y)=\frac{1}{\sqrt{2\pi}}\left[ pe^{-\frac{(y-10)^2}{2}}+(1-p)e^{-\frac{(-y-10)^2}{2}} \right]$$

        We simplify to get:

        $$\boxed{f_Y(y)=\frac{1}{\sqrt{2\pi}}\left[ pe^{-\frac{(y-10)^2}{2}}+(1-p)e^{-\frac{(y+10)^2}{2}} \right]}$$

    \end{enumerate}

    \setcounter{enumi}{4}

  \item We begin by finding the moment generating function of $X$. We can write this as:

    $$\phi_x(s)=E[e^{sX}]$$
    $$E[e^{sX}]=\sum_{x=0}^{\infty}e^{sx}\left( \frac{e^{-\alpha_1}\alpha_1^x}{x!} \right)$$

    We rearrange to get:

    $$E[e^{sX}]=e^{-\alpha_1}\sum_{x=0}^{\infty}\left( \frac{(e^{s}\alpha_1)^x}{x!} \right)$$

    We know that the series expansion for an exponential is:

    $$e^x\to\sum_{n=0}^{\infty} \frac{x^n}{n!}$$

    As such, we get:

    $$\boxed{\phi_x(s)=e^{\alpha_1(e^s-1)}}$$

    Similarly, we find the MGF of $Y$:

    $$\boxed{\phi_y(s)=e^{\alpha_2(e^s-1)}}$$

    Since $X$ and $Y$ are independent, the MGF of $Z=X+Y$ will simply be the product of the individual MGFs:

    $$\boxed{\phi_z(s)=e^{(\alpha_1+\alpha_2)(e^{s}-1)}}$$

    Note that, if we were to apply the same process to find the MGF of $Z$ as we did for $X$ and $Y$, we would still get the same answer.

    \setcounter{enumi}{6}

  \item

    \begin{enumerate}

      \item Using the moment generating function, we know that:

        $$E[Y]=\phi_y(0)'$$

        As such, we differentiate to get:

        $$\phi_y'(s)=\frac{2}{(1-s)^3}$$

        We evaluate at $s=0$, which gives us:

        $$\boxed{E[Y]=2}$$

        Similarly, we may write:

        $$E[Y^2]=\phi_y(0)''$$

        We differentiate a second time to get:

        $$\phi_y''(s)=\frac{6}{(1-s)^4}$$

        Evaluating at $s=0$, we find:

        $$\boxed{E[Y^2]=6}$$

      \item Similar to (a), we find:

        $$E[V]=\phi_w(0)'$$
        $$E[V]=\frac{3}{(1-0)^4}$$
        $$\boxed{E[V]=3}$$

        This gives us:

        $$E[W]=E[Y]+E[V]$$
        $$E[W]=2+3$$
        $$\boxed{E[W]=5}$$

        We then find:

        $$E[V^2]=\phi_w(0)''$$
        $$E[V^2]=\frac{12}{(1-0)^5}$$
        $$\boxed{E[V^2]=12}$$

        From here, we may write:

        $$\text{Var}[W]=\text{Var}[Y]+\text{Var}[V]$$

        This gives us:

        $$\text{Var}[W]=(6-2^2)+(12-3^2)$$
        $$\boxed{\text{Var}[W]=5}$$

      \item Given that $Y$ and $V$ are independent, we may take their product to find:

        $$\phi_w(s)=\frac{1}{(1-s)^5}$$

        We may observe that, accordingly, $W=\text{Erlang}(\lambda=1,k=5)$, which makes sense, given that $Y=\text{Erlang}(1,2)$ and $V=\text{Erlang}(1,3)$. As such, we get:

        $$\boxed{f_W(w)=\frac{w^4e^{-w}}{4!},\quad w\geq 0}$$

      \item We may re-express this as:

        $$W_1=\frac{1}{2}\left( Y+V \right)\to W_1=\frac{1}{2}W$$

        Accordingly, we get:

        $$E[W_1]=\frac{1}{2}E[W]$$
        $$\boxed{E[W_1]=2.5}$$

        And then:

        $$\text{Var}[W_1]=\frac{1}{4}\text{Var}[W]$$
        $$\boxed{\text{Var}[W_1]=1.25}$$

        And finally:

        $$\boxed{f_{W_1}(w_1)=\frac{2(2w_1)^4e^{-2w_1}}{4!},\quad w_1\geq0}$$

    \end{enumerate}

  \item

    \begin{enumerate}

      \item Knowing our results for a standard normal distribution, may write the moment generating function as:

        $$\phi_x(s)=E[e^{sX}]$$
        $$\boxed{\phi_x(s)=e^{\frac{s^2}{2}}}$$

      \item Given that each $X_n$ is independent of the others, we can multiply the MGFs $k$ times to write:

        $$E[e^{sV}|K=k]=e^{\frac{ks^2}{2}}$$

        From here, we use the given probabilities to write:

        $$\phi_v(s)=\sum_{k=1}^{\infty} P_K(k)e^{\frac{ks^2}{2}}$$

        This gives us:

        $$\boxed{\phi_v(s)=\frac{.8e^{\frac{s^2}{2}}}{1-.2e^{\frac{s^2}{2}}}}$$

        Before evaluating the expected value with the MGF, we may find the value to be:

        $$E[V]=E[E[V|K]]$$

        From above, we know:

        $$E[V|K]=kE[X]$$

        And since each $X$ is normally distributed, $E[X]=0$. Therefore, we may conclude:

        $$\boxed{E[V]=0}$$

        We confirm using the MGF:

        $$\phi_v(s)'=\frac{.8se^{\frac{s^2}{2}}}{\left( 1-.2e^{\frac{s^2}{2}} \right)^2}$$

        At $s=0$ this gives:

        $$\phi_v(0)'=0$$

        And therefore, we confirm:

        $$\boxed{E[V]=\phi_v'(0)=0}$$

    \end{enumerate}

  \item

    \begin{enumerate}

      \item We may observe that the given distribution is binomial, such that we may write:

        $$K_{50}=\left( \begin{matrix} 50\\ k \end{matrix} \right).7^k.3^{1-k}$$

        Accordingly, we may find:

        $$E[K_{50}]=np$$
        $$E[K_{50}]=50(.7)$$
        $$\boxed{E[K_{50}]=35\text{ video packets}}$$

      \item Again, given this is a binomial distribution, we write:

        $$\sigma_{50}=\sqrt{np(1-p)}$$
        $$\sigma_{50}=\sqrt{35(.3)}$$
        $$\boxed{\sigma_{50}=3.2404\text{ video packets}}$$

      \item We begin by standardizing to a normal variable, $Z$:

        $$Z=\frac{X-35}{3.2404}$$

        We take the limits as $30\leq X\leq 40$, such that:

        $$\frac{-5}{3.2404}\leq Z\leq \frac{5}{3.2404}$$
        $$-1.543\leq Z\leq 1.543$$

        We can find this probability by first finding:

        $$P[Z\geq -1.543]=.9386$$

        We then subtract the upper bound:

        $$P[30\leq X\leq40]=.9386-P[Z\geq 1.543]$$
        $$P[30\leq X\leq40]=.9386-.061415$$
        $$P[30\leq X\leq40]=.8772$$

      \item We begin by using continuity correction to write:

        $$P[a\leq K_n\leq b]\approx P\left[ \frac{a-\mu}{\sigma}\leq Z\leq \frac{b-\mu}{\sigma} \right]$$

        This gives us:

        $$P[30\leq K_n\leq 50]\approx P\left[ 29.5\leq K_n\leq 40.5 \right]$$
        $$P[30\leq K_n\leq 50]\approx P\left[ -1.6973\leq Z\leq 1.6973 \right]$$

        We can use the inverse normal operation to find:

        $$P[30\leq K_n\leq 50]\approx P[Z\leq 1.6973]-P[Z\leq -1.6973]$$
        $$P[30\leq K_n\leq 50]\approx .955-.045$$
        $$\boxed{P[30\leq K_n\leq 50]\approx .91}$$

    \end{enumerate}

    \setcounter{enumi}{10}

  \item

    \begin{enumerate}

      \item We know that the expectation value for a single test is given by the inverse of its probability, or:

        $$E[X_n]=\frac{1}{p}$$

        Given that we sum together a series consisting of $n$ of these geometric distributions, we can find:

        $$E[K_n]=E[X_1]+E[X_2]+\cdots+E[X_n]=\frac{n}{p}$$

        This gives us:

        $$E[K_n]=\frac{900}{.9}$$
        $$\boxed{E[K_n]=1000\text{ tests}}$$

      \item Similarly, the variance of $K_n$ may be written as:

        $$\text{Var}[K_n]=n\text{Var}[X_n]$$

        We may find:

        $$\text{Var}[X_n]=\frac{1-p}{p^2}$$
        $$\text{Var}[X_n]=\frac{.1}{.9^2}$$
        $$\boxed{\text{Var}[X_n]=.1235\text{ tests}^2}$$

        We then multiply by $n$ to find:

        $$\boxed{\text{Var}[K_n]=111.11\text{ tests}^2}$$

      \item Using the CLT for \underline{at least} 1000 tests, we write:

        $$Z=\frac{1000-\mu}{\sigma}=\frac{1000-1000}{\sqrt{111.11}}$$
        $$Z=0$$

        From here, we apply a normal distribution to write:

        $$\boxed{P[Z\geq 0]=.5}$$

    \end{enumerate}

  \item We can begin by finding:

    $$E[V]=20-10E[W^3]$$

    We can find this by using:

    $$E[W^3]=\int_{-1}^1 \frac{w^3}{2}\,dw$$
    $$E[W^3]=\frac{w^4}{8}\Big|_{-1}^1$$
    $$E[W^3]=0$$

    This gives us:

    $$E[V]=20$$

    Similarly, we write:

    $$\text{Var}[V]=100\text{Var}[W^3]$$
    $$\text{Var}[V]=100\int_{-1}^1 (w^3)^2\cdot\frac{1}{2}\,dw$$
    $$\text{Var}[V]=100\int_{-1}^1 \frac{w^6}{2}\,dw$$
    $$\text{Var}[V]=100\left[\frac{w^7}{14}\right]\Big|_{-1}^1$$
    $$\text{Var}[V]=100/7\approx 14.285$$

    As such, we may write expressions for $X$ as:

    $$E[X]=30\cdot E[V]$$
    $$\sigma_X=\sqrt{30\cdot \text{Var}[V]}$$

    These give us:

    $$E[X]=600\text{ Mb}$$
    $$\sigma_X=20.73\text{ Mb}$$

    We then want to find:

    $$P\left[ \frac{S_{30}}{6}\geq95 \right]\to P\left[ Z\geq \frac{X-\mu}{\sigma}\right]$$
    $$P\left[ S_{30}\geq 570 \right]\to P\left[ Z\geq \frac{X-\mu}{\sigma}\right]$$

    We apply the central limit theorem to get:

    $$Z=\frac{570-600}{20.73}$$
    $$Z=-1.4472$$

    From here, we find:

    $$P\left[ S_{30}\geq 570 \right]= 1-P[Z< -1.4472]$$
    $$\boxed{P\left[ X\geq 95 \right]\approx .9261}$$

\end{enumerate}

\end{document}

