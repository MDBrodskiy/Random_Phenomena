%%%%%%%%%%%%%%%%%%%%%%%%%%%%%%%%%%%%%%%%%%%%%%%%%%%%%%%%%%%%%%%%%%%%%%%%%%%%%%%%%%%%%%%%%%%%%%%%%%%%%%%%%%%%%%%%%%%%%%%%%%%%%%%%%%%%%%%%%%%%%%%%%%%%%%%%%%%%%%%%%%%
% Written By Michael Brodskiy
% Class: Analysis of Random Phenomena
% Professor: I. Salama
%%%%%%%%%%%%%%%%%%%%%%%%%%%%%%%%%%%%%%%%%%%%%%%%%%%%%%%%%%%%%%%%%%%%%%%%%%%%%%%%%%%%%%%%%%%%%%%%%%%%%%%%%%%%%%%%%%%%%%%%%%%%%%%%%%%%%%%%%%%%%%%%%%%%%%%%%%%%%%%%%%%

\documentclass[12pt]{article} 
\usepackage{alphalph}
\usepackage[utf8]{inputenc}
\usepackage[russian,english]{babel}
\usepackage{titling}
\usepackage{float}
\usepackage{amsmath}
\usepackage{graphicx}
\usepackage{enumitem}
\usepackage{amssymb}
\usepackage[super]{nth}
\usepackage{everysel}
\usepackage{ragged2e}
\usepackage{geometry}
\usepackage{multicol}
\usepackage{fancyhdr}
\usepackage{cancel}
\usepackage{siunitx}
\usepackage{physics}
\usepackage{tikz}
\usepackage{mathdots}
\usepackage{yhmath}
\usepackage{cancel}
\usepackage{color}
\usepackage{xcolor}
\usepackage{colortbl}
\usepackage{array}
\usepackage{multirow}
\usepackage{gensymb}
\usepackage{tabularx}
\usepackage{extarrows}
\usepackage{booktabs}
\usepackage{lastpage}
\usetikzlibrary{fadings}
\usetikzlibrary{patterns}
\usetikzlibrary{shadows.blur}
\usetikzlibrary{shapes}

\geometry{top=1.0in,bottom=1.0in,left=1.0in,right=1.0in}
\newcommand{\subtitle}[1]{%
  \posttitle{%
    \par\end{center}
    \begin{center}\large#1\end{center}
    \vskip0.5em}%

}
\usepackage{hyperref}
\hypersetup{
colorlinks=true,
linkcolor=blue,
filecolor=magenta,      
urlcolor=blue,
citecolor=blue,
}


\title{Final}
\date{\today}
\author{Michael Brodskiy\\ \small Professor: I. Salama}

\begin{document}

\maketitle

\begin{justifying}
  \textbf{On my honor, I pledge to uphold the values of honesty, integrity, and respect that are expected of me as a Northeastern student.}\\

 Northeastern University holds all students accountable for the honest completion of examinations, tests, papers, projects, and assignments. students are obligated to approach these tasks with unwavering integrity.\\

Students should complete this quiz independently, relying solely on their own efforts and understanding. The work submitted will be a true reflection of their individual endeavor. Please  answer all components of the questions and provide comprehensive details of your work.

\end{justifying}

\vspace{15pt}

\begin{enumerate}

  \item

    \begin{enumerate}

      \item First, we may write the autocorrelation function as:

        $$R_{XX}[n,k]=E[X_nX_{n+k}]$$

        Given their independence, for $k\neq 0$ we may write:

        $$R_{XX}[n,k]=E[X_n]E[X_{n+k}]$$

        We are given the expectation value of $X$, which allows use to write:

        $$R_{XX}[n,k]=(\mu_x)^2$$
        $$R_{XX}[n,k]=(2)^2$$

        When we take $k\to0$, we find that the variance component does not vanish such that:

        $$R_{XX}[n,0]=E[X_n^2]$$
        $$R_{XX}[n,0]=\text{Var}(X_n)+(E[X_n])^2$$
        $$R_{XX}[n,0]=2^2+(2)^2$$
        $$\boxed{R_{XX}[n,k]=\left\{\begin{array}{ll} 4, & k\neq0\\8, & k=0\end{array}}$$

        The autocovariance may be written as:

        $$C_{XX}[n,k]=R_{XX}[n,k]-E[X_n]E[X_{n+k}]$$

        We may notice that we simply subtract 4 from the autocorrelation to get:

        $$\boxed{C_{XX}[n,k]=\left\{\begin{array}{ll} 0, & k\neq0\\4, & k=0\end{array}}$$

      \item We may express the expected power as:

        $$E[X_n^2]=\text{Var}(X_n)+(E[X_n])^2$$

        We may see that this is equivalent to the $k=0$ value of the autocorrelation. Thus, we get:

        $$\boxed{E[X_n^2]=8}$$

      \item We can express the expected value of the output sequence as:

        $$E[Y_n]=\frac{1}{2}(E[X_n]-E[X_{n-2}])$$

        Since each $X_n$ has the same mean, we find:

        $$\boxed{E[Y_n]=0}$$

      \item We may rewrite the expression as:

        $$R_{YY}[n,k]=E[Y_nY_{n+k}]$$
        $$R_{YY}[n,k]=\frac{1}{4}E[(X_n-X_{n-2})(X_{n+k}-X_{n+k-2})]$$

        We distribute to get:

        $$R_{YY}[n,k]=\frac{1}{4}E[X_nX_{n+k}-X_{n-2}X_{n+k}-X_nX_{n+k-2}+X_{n-2}X_{n+k-2}]$$

        We first take $k\to0$ to get:

        $$R_{YY}[n,0]=\frac{1}{4}E[X_n^2-2X_{n-2}X_{n}+X_{n-2}^2]$$

        This gives us:

        $$R_{YY}[n,0]=\frac{1}{4}(8-2(2)(2)+8)$$
        $$R_{YY}[n,0]=2$$

        Next, we take $k\to\pm2$ to get:

        $$R_{YY}[n,\pm2]=\frac{1}{4}E[X_nX_{n\pm2}-X_{n-2}X_{n\pm2}-X_nX_{n\pm2-2}+X_{n-2}X_{n\pm2-2}]$$
        $$R_{YY}[n,\pm]=\frac{1}{4}(4-4-8+4)$$
        $$R_{YY}[n,\pm]=-1$$

        When $k\neq0$, we find:

        $$R_{YY}[n,k]=\frac{1}{4}(2-2-2+2)$$
        $$R_{YY}[n,k]=0$$

        As such, we conclude:

        $$\boxed{R_{YY}[n,k]=\left\{\begin{array}{ll} 0, & k\neq0,\pm2\\ -1, & k=\pm2\\2, & k=0\end{array}}$$

        The autocovariance may be written as:

        $$C_{YY}[n,k]=R_{YY}[n,k]-E[Y_nY_{n+k}]$$

        Since the expectation value of $Y$ is zero, we get:

        $$C_{YY}[n,k]=R_{YY}[n,k]$$

        $$\boxed{C_{YY}[n,k]=\left\{\begin{array}{ll} 0, & k\neq0,\pm2\\ -1, & k=\pm2\\2, & k=0\end{array}}$$

        \item The sequence $Y_n$ \underline{is wide-sense stationary}, since its mean is constant for all indices (it is always zero) and the autocorrelation is dependent solely on the index shift, $k$ ($R_{YY}[n,k]=R_{YY}[n-k]$)

        \item The \underline{components of $Y_n$ are  not uncorrelated}. We may conclude this since, as shown in part (d), the autocorrelation is nonzero for at least one value of $k\neq0$ (when $k=\pm2$ to be precise)

      \item 

        \begin{enumerate}

          \item We begin by finding the mean. We do this by writing:

            $$E[W_n]=E[B_nX_n]$$

            Since $B_n$ and $X_n$ are independent, we get:

            $$E[W_n]=E[B_n]E[X_n]$$

            The mean for $X_n$ is given and, by the properties of Bernoulli random variables, we know $E[B_n]=p$. Thus, we get:

            $$\mu_W=E[W_n]=(p)(2)$$
            $$\boxed{\mu_W=(p)(2)}$$

            We can find the autocorrelation by writing:

            $$R_{WW}[n,k]=E[W_nW_{n+k}]$$

            This gives us:

            $$R_{WW}[n,k]=E[B_nX_nB_{n+k}X_{n+k}]$$

            Taking $k\to0$ gives us:

            $$R_{WW}[n,k]=E[B_n^2X_n^2]$$

            Given their independence, we get:

            $$R_{WW}[n,0]=E[B_n^2]E[X_n^2]$$
            $$R_{WW}[n,0]=(p)(8)$$
            $$R_{WW}[n,0]=8p$$

            We then find, that for $k\neq0$, we have:

            $$R_{WW}[n,k]=E[B_n]E[X_n]E[X_{n+k}]E[B_{n+k}]$$
            $$R_{WW}[n,k]=(p)(2)(2)(p)$$
            $$R_{WW}[n,k]=4p^2$$

            Thus, we write:

            $$\boxed{R_{WW}[n,k]=\left\{\begin{array}{ll} 4p^2, & k\neq0\\ 8p, & k=0\end{array}}$$

            We may write the autocovariance as:

            $$C_{WW}[k]=R_{WW}[k]-(E[W_k])^2$$

            If we take $k\to0$, we get:
            
            $$C_{WW}[0]=R_{WW}[0]-(E[W_0])^2$$
            $$C_{WW}[0]=8p-(2p)^2$$

            For $k\neq0$, we find:

            $$C_{WW}[k]=R_{WW}[k]-4p^2$$
            $$C_{WW}[k]=4p^2-4p^2$$
            $$C_{WW}[k]=0$$

            Thus, we write:

            $$\boxed{C_{WW}[k]=\left\{\begin{array}{ll} 8p-4p^2, & k=0\\ 0, & k\neq0\end{array}}$$

            Since the autocorrelation is zero for all values when $k\neq0$, we may conclude that the components of $W_n$ \underline{are uncorrelated}

          \item Similar to how we did it previously, we get:

            $$R_{WX}[n,k]=E[W_nX_{n+k}]$$

            We expand this to get:

            $$R_{WX}[n,k]=E[B_nX_nX_{n+k}]$$

            Taking $k\to0$, we get:

            $$R_{WX}[n,0]=E[B_nX_n^2]$$
            $$R_{WX}[n,0]=8p$$

            And then for $k\neq0$, we get:

            $$R_{WX}[n,k]=E[B_nX_nX_{n+k}]$$
            $$R_{WX}[n,k]=4p$$

            Thus, we write:

            $$\boxed{R_{WX}[n,k]=\left\{\begin{array}{ll} 4p, & k\neq0\\ 8p, & k=0\end{array}}$$

            We continue to get:

            $$C_{WX}[k]=R_{WX}[k]-E[W_k]E[X_{n+k}]$$

            This gives us (for $k\to0$):

            $$C_{WX}[0]=R_{WX}[0]-(2p)(2)$$
            $$C_{WX}[0]=8p-(4p)$$
            $$C_{WX}[0]=4p$$

            And for $k\neq0$:

            $$C_{WX}[k]=R_{WX}[k]-E[W_k]E[X_k]$$
            $$C_{WX}[k]=4p-4p$$
            $$C_{WX}[k]=0$$

            Thus, we conclude:

            $$\boxed{C_{WX}[k]=\left\{\begin{array}{ll} 4p, & k=0\\ 0, & k\neq0\end{array}}$$

          \item Finally, we compare $Y$ and $W$ by writing:

            $$R_{YW}[n,k]=E[Y_nW_{n+k}]$$

            We expand to get:

            $$R_{YW}[n,k]=\frac{1}{2}E[(X_n-X_{n-2})B_{n+k}X_{n+k}]$$
            $$R_{YW}[n,k]=\frac{1}{2}E[X_nB_{n+k}X_{n+k}-B_{n+k}X_{n+k}X_{n-2}]$$

            Taking $k\to0$, we see:

            $$R_{YW}[n,0]=\frac{1}{2}E[B_{n}X_{n}^2-B_{n}X_{n}X_{n-2}]$$
            $$R_{YW}[n,0]=\frac{1}{2}[(8)(p)-(p)(2)(2)]$$
            $$R_{YW}[n,0]=2p$$

            For $k=2$, we see:

            $$R_{YW}[n,2]=\frac{1}{2}E[X_nB_{n+2}X_{n+2}-B_{n+2}X_{n+2}X_{n-2}]$$
            $$R_{YW}[n,2]=\frac{1}{2}[(2)(p)(2)-(p)(2)(2)]$$
            $$R_{YW}[n,2]=0$$

            Furthermore, for $k=-2$, we find:

            $$R_{YW}[n,-2]=\frac{1}{2}E[X_nB_{n-2}X_{n-2}-B_{n-2}X_{n-2}^2]$$
            $$R_{YW}[n,-2]=\frac{1}{2}[(2)(p)(2)-(p)(8)]$$
            $$R_{YW}[n,-2]=-2p$$

            For $k\neq0,-2,2$, this is:

            $$R_{YW}[n,k]=\frac{1}{2}E[X_nB_{n+k}X_{n+k}-B_{n+k}X_{n+k}X_{n-2}]$$
            $$R_{YW}[n,k]=\frac{1}{2}[(2)(2)(p)-(p)(2)(2)]$$
            $$R_{YW}[n,k]=0$$

            As such, we write:

            $$\boxed{R_{YW}[n,k]=\left\{\begin{array}{ll} 0, & k\neq0,-2\\ 2p, & k=0\\-2p, & k=-2\end{array}}$$

            Given that the expectation value of $Y$ is zero, we get:

            $$C_{YW}[k]=R_{YW}[k]-E[Y_n]E[W_{n+k}]$$
            $$C_{YW}[k]=R_{YW}[k]$$
            $$\boxed{C_{YW}[k]=\left\{\begin{array}{ll} 0, & k\neq0,-2\\ 2p, & k=0\\-2p, & k=-2\end{array}}$$

        \end{enumerate}

    \end{enumerate}

  \item

    \begin{enumerate}

      \item First and foremost, we know that, for a Gaussian distribution, we have:

        $$\boxed{E[X(t)]=0}$$

        Now, we may find the power by using the autocorrelation and taking $\tau\to0$:

        $$E[X^2(t)]=25e^{-20000\tau^2}$$
        $$E[X^2(t)]=25e^{-20000(0)^2}$$
        $$\boxed{E[X^2(t)]=25}$$

      \item We can find this probability by converting to a $Z$-score:

        $$Z=\frac{X(1[\si{\milli\second}])}{\sigma}$$

        This gives us:

        $$P\left[ Z<\frac{2}{\sqrt{25}} \right]=P\left[\frac{X(1[\si{\milli\second}])}{5}<.4\right]$$
        $$\boxed{P\left[ Z<.4 \right]=.6554}$$

      \item We may begin by expanding 

      \item 

      \item 

      \item 

    \end{enumerate}

  \item

    \begin{enumerate}

      \item We may begin by writing:

        $$E[X(t)]=AE[\cos(\omega t+\theta)]$$

        We use the cosine identity:

        $$\cos(\omega t+\theta)=\cos(\omega t)\cos(\theta)-\sin(\omega t)\sin(\theta)$$

        To get:

        $$E[X(t)]=A\left( E[\cos(\omega t)]E[\cos(\theta)]-E[\sin(\omega t)]E[\sin(\theta)] \right)$$

        Since $\theta$ is a uniform distribution from $-\pi$ to $\pi$, the $\theta$-dependent terms become zero, which gives us:

        $$E[X(t)]=A(0-0)$$
        $$\boxed{E[X(t)]=0}$$

      \item We can write the autocorrelation function as:

        $$R_{XX}(t,\tau)=E[X(t)X(t+\tau)]$$

        We substitute our values to get:

        $$R_{XX}(t,\tau)=A^2E[\cos(\omega t+\theta)\cos(\omega(t+\tau)+\theta)]$$

        Using the identity:

        $$\cos(a)\cos(b)=\frac{1}{2}\left[ \cos(a-b)+\cos(a+b) \right]$$

        This gives us:

        $$E[X(t)X(t+\tau)]=\frac{A^2}{2}E\left[ \cos(\omega \tau)+\cos(2\omega t+\omega\tau+2\theta) \right]$$

        We may observe that, since the second term is dependent on $\theta$, it cancels out, which leaves us with:

        $$E[X(t)X(t+\tau)]=\frac{A^2}{2}E\left[ \cos(\omega \tau)\right]$$

        We proceed to calculate the expectation value as:

        $$E[X(t)X(t+\tau)]=\frac{A^2}{2}\int_{-\omega_o}^{\omega_o}\cos(\omega\tau)\,d\omega$$

        We see that this results in the sinc function, which lets us write:

        $$\boxed{R_{XX}(t,\tau)=\frac{A^2}{2}\text{sinc}\left( \frac{\omega_o\tau}{\pi} \right)}$$

      \item We may conclude that $X(t)$ \underline{is wide-sense stationary} since the mean is constant (0, per part a), and the autocorrelation depends only on the time difference, $\tau$ (as per part b)

      \item We find the variance as:

        $$\text{Var}(X(t))=R_{XX}(X(0))$$
        $$\text{Var}(X(t))=\frac{A^2}{2} \text{sinc}(0)$$
        $$\boxed{\text{Var}(X(t))=\frac{A^2}{2}}$$

    \end{enumerate}

\end{enumerate}

\end{document}

