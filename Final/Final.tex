%%%%%%%%%%%%%%%%%%%%%%%%%%%%%%%%%%%%%%%%%%%%%%%%%%%%%%%%%%%%%%%%%%%%%%%%%%%%%%%%%%%%%%%%%%%%%%%%%%%%%%%%%%%%%%%%%%%%%%%%%%%%%%%%%%%%%%%%%%%%%%%%%%%%%%%%%%%%%%%%%%%
% Written By Michael Brodskiy
% Class: Analysis of Random Phenomena
% Professor: I. Salama
%%%%%%%%%%%%%%%%%%%%%%%%%%%%%%%%%%%%%%%%%%%%%%%%%%%%%%%%%%%%%%%%%%%%%%%%%%%%%%%%%%%%%%%%%%%%%%%%%%%%%%%%%%%%%%%%%%%%%%%%%%%%%%%%%%%%%%%%%%%%%%%%%%%%%%%%%%%%%%%%%%%

\documentclass[12pt]{article} 
\usepackage{alphalph}
\usepackage[utf8]{inputenc}
\usepackage[russian,english]{babel}
\usepackage{titling}
\usepackage{float}
\usepackage{amsmath}
\usepackage{graphicx}
\usepackage{enumitem}
\usepackage{amssymb}
\usepackage[super]{nth}
\usepackage{everysel}
\usepackage{ragged2e}
\usepackage{geometry}
\usepackage{multicol}
\usepackage{fancyhdr}
\usepackage{cancel}
\usepackage{siunitx}
\usepackage{physics}
\usepackage{tikz}
\usepackage{mathdots}
\usepackage{yhmath}
\usepackage{cancel}
\usepackage{color}
\usepackage{xcolor}
\usepackage{colortbl}
\usepackage{array}
\usepackage{multirow}
\usepackage{gensymb}
\usepackage{tabularx}
\usepackage{extarrows}
\usepackage{booktabs}
\usepackage{lastpage}
\usetikzlibrary{fadings}
\usetikzlibrary{patterns}
\usetikzlibrary{shadows.blur}
\usetikzlibrary{shapes}

\geometry{top=1.0in,bottom=1.0in,left=1.0in,right=1.0in}
\newcommand{\subtitle}[1]{%
  \posttitle{%
    \par\end{center}
    \begin{center}\large#1\end{center}
    \vskip0.5em}%

}
\usepackage{hyperref}
\hypersetup{
colorlinks=true,
linkcolor=blue,
filecolor=magenta,      
urlcolor=blue,
citecolor=blue,
}


\title{Final}
\date{\today}
\author{Michael Brodskiy\\ \small Professor: I. Salama}

\begin{document}

\maketitle

\begin{enumerate}

  \item

    \begin{enumerate}

      \item First, we may write the autocorrelation function as:

        $$R_{XX}[n,k]=E[X_nX_{n+k}]$$

        Given their independence, we may write:

        $$R_{XX}[n,k]=E[X_n]E[X_{n+k}]$$

        We are given the expectation value of $X$, which allows use to write:

        $$R_{XX}[n,k]=(\mu_x)^2$$
        $$R_{XX}[n,k]=(2)^2$$
        $$\boxed{R_{XX}[n,k]=4}$$

        The autocovariance may be written as:

        $$C_{XX}[n,k]=R_{XX}[n,k]-E[X_n]E[X_{n+k}]$$
        $$\boxed{C_{XX}[n,k]=0,\quad k\neq0}$$

        Since the functions are i.i.d., we find that the autocovariance is zero, as the autocorrelation is equivalent to the product of the means.

      \item We may express the expected power as:

        $$E[X_n^2]=\text{Var}(X_n)+(E[X_n])^2$$

        We may calculate the variance for an exponential distribution as:

        $$\text{Var}(\text{Exp}(\mu=2))=(\mu)^2$$
        $$\text{Var}(X_n)=4$$

        As such, we get:

        $$E[X_n^2]=4+4$$
        $$\boxed{E[X_n^2]=8}$$

      \item We can express the expected value of the output sequence as:

        $$E[Y_n]=\frac{1}{2}(E[X_n]-E[X_{n-2}])$$

        Since each $X_n$ has the same mean, we find:

        $$\boxed{E[Y_n]=0}$$

      \item We may rewrite the expression as:

        $$R_{YY}[n,k]=E[Y_nY_{n+k}]$$
        $$R_{YY}[n,k]=\frac{1}{4}E[(X_n-X_{n-2})(X_{n+k}-X_{n+k-2})]$$

        Since the mean for every $X$ is the same, we can drop the subscripts to get:

        $$R_{YY}[n,k]=\frac{1}{4}E[(X-X)(X-X)]$$
        $$R_{YY}[n,k]=\frac{1}{4}E[(0)(0)]$$
        $$\boxed{R_{YY}[n,k]=0}$$

        The autocovariance may be written as:

        $$C_{YY}[n,k]=R_{YY}[n,k]-E[Y_nY_{n+k}]$$

        We may observe that both of the terms defined above are equivalent to zero, which leads us to conclude:

        $$\boxed{C_{YY}[n,k]=0}$$

      \item The sequence $Y_n$ \underline{is wide-sense stationary}, since its mean is constant for all indices and the autocorrelation is dependent solely on the time shift ($R_{YY}[n,k]=R_{YY}[n-k]$)

      \item The \underline{components of $Y_n$ are uncorrelated}. We may conclude this since, for the entire sequence, $R_{YY}[n,k]=0$.

      \item 

        \begin{enumerate}

          \item 

          \item 

          \item 

        \end{enumerate}

    \end{enumerate}

  \item

    \begin{enumerate}

      \item First and foremost, we know that, for a Gaussian distribution, we have:

        $$\boxed{E[X(t)]=0}$$

        Now, we may find the power by using the autocorrelation and taking $\tau\to0$:

        $$E[X^2(t)]=25e^{-20000\tau^2}$$
        $$E[X^2(t)]=25e^{-20000(0)^2}$$
        $$\boxed{E[X^2(t)]=25}$$

      \item We can find this probability by converting to a $Z$-score:

        $$Z=\frac{X(1[\si{\milli\second}])}{\sigma}$$

        This gives us:

        $$P\left[ Z<\frac{2}{\sqrt{25}} \right]=P\left[\frac{X(1[\si{\milli\second}])}{5}<.4\right]$$
        $$\boxed{P\left[ Z<.4 \right]=.6554}$$

      \item We may begin by expanding 

      \item 

      \item 

      \item 

    \end{enumerate}

  \item

    \begin{enumerate}

      \item 

      \item 

      \item 

      \item 

    \end{enumerate}

\end{enumerate}

\end{document}

